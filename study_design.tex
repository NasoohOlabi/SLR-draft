\section{Literature review methodology}\label{sec:design}

\subsection{Research questions}
Here are the research questions addressed in this SLR:
\begin{itemize}
    \item What is the state of published literature on steganographic techniques that leverage large language models (LLMs)?
    \item In which applications are steganographic techniques with LLMs being explored?
    \item What metrics and evaluation methods are used to assess the performance of steganographic techniques in LLMs, focusing on factors like capacity, security, and contextual compatibility?
    \item How are external knowledge sources (semantic resources) integrated into steganographic techniques with LLMs to enhance capacity or contextual relevance?
    \item What are the limitations and trade-offs associated with current steganographic techniques using LLMs, particularly concerning security, capacity, and contextual compatibility?
    \item What are the potential future research directions in steganography with LLMs, considering emerging trends and identified gaps in the literature?
\end{itemize}

\subsection{Search query string}
We used the following search query string for our initial literature search:
\begin{verbatim}
(steganography or watermark or "Information Hiding")
 and ("Large Language Model" or LLM or BERT or LAMA or GPT)
\end{verbatim}

\subsection{Study selection and quality assessment}
We established the following inclusion and exclusion criteria for study selection:

\subsubsection{Inclusion Criteria}
\begin{itemize}
    \item \textbf{Full Text Access}: Studies for which the full text is available.
    \item \textbf{Language}: Publications written in English.
    \item \textbf{Peer-reviewed}: Articles published in peer-reviewed journals, conferences, or workshops.
    \item \textbf{Publication Date}: Studies published from 2018 onwards, to focus on recent advancements in LLMs.
    \item \textbf{Relevance}: Studies directly addressing steganography, watermarking, or information hiding techniques that utilize or are significantly impacted by Large Language Models (LLMs), BERT, LAMA, or GPT architectures.
    \item \textbf{Research Type}: Empirical studies, surveys, reviews, and theoretical contributions.
\end{itemize}

\subsubsection{Exclusion Criteria}
\begin{itemize}
    \item \textbf{Duplicated Studies}: Multiple publications reporting the same study will be excluded, with the most complete or recent version retained.
    \item \textbf{Incomplete or Abstract-only}: Studies for which only an abstract is available or the full text is incomplete.
    \item \textbf{Irrelevant Studies}: Publications not directly related to steganography with LLMs.
    \item \textbf{Non-English Publications}: Studies not published in English.
    \item \textbf{Non-peer-reviewed Sources}: Preprints, dissertations, theses, books, and book chapters (unless they are extended versions of peer-reviewed conference papers).
\end{itemize}

\subsection{Bibliometric analysis}
Briefly note if snowballing was used for additional sources.

% The following subsections are not explicitly mentioned in the user's requested TOC, but are part of the original study_design.tex. I will keep them for now, but they might need to be merged or removed based on further instructions.
% \subsection{Data Extraction}
% Describe how data was extracted from selected studies.

% \subsection{Data Synthesis}
% Summarize the approach for synthesizing extracted data.

% \subsection{Study Replicability}
% List materials/data made available for replicability.
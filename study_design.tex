\section{Study Design}\label{sec:design}
\label{sec:design}

\subsection{Research Goal}
State the main research goal (e.g., characterize LLM-based linguistic steganography).

\subsection{Research Questions}
Here are the research questions addressed in this SLR:
\begin{itemize}
    \item What is the state of published literature on steganographic techniques that leverage large language models (LLMs)?
    \item In which applications are steganographic techniques with LLMs being explored?
    \item What metrics and evaluation methods are used to assess the performance of steganographic techniques in LLMs, focusing on factors like capacity, security, and contextual compatibility?
    \item How are external knowledge sources (semantic resources) integrated into steganographic techniques with LLMs to enhance capacity or contextual relevance?
    \item What are the limitations and trade-offs associated with current steganographic techniques using LLMs, particularly concerning security, capacity, and contextual compatibility?
    \item What are the potential future research directions in steganography with LLMs, considering emerging trends and identified gaps in the literature?
\end{itemize}

\subsection{Initial search}
Describe the initial literature search process.

\subsection{Application of selection criteria}
Summarize inclusion/exclusion criteria for study selection.

\subsection{Snowballing}
Briefly note if snowballing was used for additional sources.

\subsection{Data Extraction}
Describe how data was extracted from selected studies.

\subsection{Data Synthesis}
Summarize the approach for synthesizing extracted data.

\subsection{Study Replicability}
List materials/data made available for replicability.
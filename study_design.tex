\section{Study Design}\label{sec:design}
\label{sec:design}
% Section dedicated to document your study was carried out

\subsection{Research Goal}
\todo{What is the research goal of this study should be documented in this section. This can follow the GQM approach, as described in the following sentence, which NEEDS TO BE REPHRASED IN YOUR OWN WORDS if you want to include it in your study.}

% The research was designed with the goal of characterizing comprehensively the current state of the art of ATD identification research. More specifically, by following the Goal-Question-Metric approach~\cite{gqm}, our goal can be formalized as follows:

% \begin{table}[!htbp]
%     \begin{tabular}{ p{1.2cm} p{6.5cm} }
%     \textit{Purpose} &  \\ 
%     \textit{Issue} &  \\
%     \textit{Object} &  \\ 
%     \textit{Viewpoint} &  \\ 
%     \end{tabular}
%     \label{tab:gqm}
% \end{table}


\subsection{Research Questions}
\label{sec:questions}
\todo{In this subsection, the research questions underlying your study should be reported}

\subsection{Initial search}
\todo{How was the initial search for literature carried out should be reported in this section}


\subsection{Application of selection criteria}
\todo{ In this section, the selection criteria, in terms of inclusion and inclusion criteria should be documented. See examples below, which should be adapted to your specific case and, in any case, *rephrased in our own words*}
% \begin{enumerate}
% \item[{I}1-] Studies focusing on TD identification in software-intensive systems. This inclusion criterion is utilized to select exclusively studies considering TD. 
%   \item[{I}2-] Studies focusing on the architecture of software-intensive systems. This inclusion criterion is utilized to filter out studies considering other levels of abstraction, such as specific code implementation details. 
% \item[{I}3-] Studies presenting or using a technique aimed to the identification of ATD in software-intensive systems. With this inclusion criteria, we ensure that only papers discussing the identification of ATD are included.
% \item[{E}1-]Secondary or tertiary studies (e.g., systematic literature reviews, surveys, etc.). This exclusion criterion is adopted in order to exclude studies which do not report the desired level of detail of ATD identification techniques.

% \item[{E}2-] Studies in the form of editorials and tutorial, short papers, and poster, as they are deemed to not provide the required level of detail and information.
  
% \item[{E}3-] Studies that have not been published in English language, as their analysis would result to be too time consuming.
  
% \item[{E}4-] Studies that have not been peer reviewed, in order to ensure the high quality of the studies considered.

% \item[{E}5-] Duplicate papers or extensions of already included papers, in order to avoid possible threats to conclusion validity.

% \item[{E}6-] Papers that are not available, as we cannot inspect them. 
% \end{enumerate}

\subsection{Snowballing}
\todo{If a snowballing approach was used for the literature review, its details should be documented in this subsection}

\subsection{Data Extraction}
\todo{Report in this section the data extraction followed to gather the data for the study (e.g., what process did you followed to gather the data in the companion data extraction spreadsheet?)}

\subsection{Data Synthesis}
\todo{What approach did you followed to carry out the data synthesis process and summarize the data extracted from the primary studies?}

\subsection{Study Replicability}
\todo{To ensure the replicability of the study, you should document in this section the data you make available (e.g. via Google Drive) to replicate the findings, this include:
* the research protocol
* the complete list of primary studies
* the parameters composing the potential classification framework you built 
* the raw extracted data of each selection phase}
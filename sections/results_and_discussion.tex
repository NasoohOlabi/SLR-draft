
\renewcommand{\arraystretch}{1.3}
\begin{longtable}{p{0.12\linewidth}p{0.12\linewidth}p{0.12\linewidth}p{0.18\linewidth}p{0.12\linewidth}p{0.12\linewidth}p{0.12\linewidth}}
\caption{Summary of Results from Reviewed Papers} \\
\toprule

Paper & Llm & Dataset & Result & Context Aware & Categ Context & Representation Context \\
\midrule

\endfirsthead

\multicolumn{7}{c}{\bfseries \tablename\ \thetable{} -- continued from previous page} \\
\toprule
Paper & Llm & Dataset & Result & Context Aware & Categ Context & Representation Context \\
\midrule

\endhead

\midrule
\multicolumn{7}{r}{Continued on next page} \\
\endfoot

\bottomrule
\endlastfoot

VAE-Stega: linguistic steganography based on va... \cite{yang2020vae} & BERTBASE (BERT-LSTM) (LSTM-LSTM) model was trained from scratch & Twitter (2.6M sentences) IMDB (1.2M sentences) preprocessed & PPL: 28.879, \ensuremath{\Delta}MP: 0.242, KLD: 3.302, JSD: 10.411, Acc: 0.600, R: 0.616 & non-explicit & pre-text & text \\

General framework for reversible data hiding in... \cite{zheng2022general} & BERTBase & BookCorpus & BPW=0.5335 F1=0.9402 PPL=134.2199 & non-explicit & pre-text & text \\

Co-stega: Collaborative linguistic steganograph... \cite{liao2024co} & Llama-2-7B-chat, GPT-2 (fine-tuned), Llama-2-13B & Tweet dataset (for GPT-2 fine-tuning), Twitter (real-time testing) & SR1: 60.87\%, SR2: 98.55\%, Gen. Capacity: 44.91 bits, Entropy: 49.21 bits, BPW: 2.31, PPL: 16.75, SimCSE: 0.69 & explicit & Social Media & text \\

Joint linguistic steganography with BERT masked... \cite{ding2023joint} & LSTM + attention for temporal context. GAT for spatial token relationships. BERT MLM for deep semantic context in substitution. & OPUS & PPL=13.917 KLD=2.904 SIM=0.812 ER=0.365 (BN=2) Best Acc=0.575 (BERT classifier) FLOPs=1.834G & explicit & pre-text & text \\

Discop: Provably secure steganography in practi... & GPT-2 & IMDB & p=1.00 Total Time (seconds)=362.63 Ave Time ↓ (seconds/bit)=6.29E-03 Ave KLD ↓ (bits/token)=0 Max KLD ↓ (bits/token)=0 Capacity (bits/token)=5.76 E... & non-explicit & tuning + pretext & text \\

Generative text steganography with large langua... \cite{wu2024generative} & Any & [Not specified] & Length: 13.333 (words). BPW: 5.93 bpw PPL: 165.76. Semantic Similarity (SS): 0.5881 LS-CNN Acc: 51.55\%. BiLSTM-Dense Acc: 49.20\%. Bert-FT Acc: 50... & explicit & [Not specified] & [Not specified] \\

Meteor: Cryptographically secure steganography ... \cite{kaptchuk2021meteor} & GPT-2 & Hutter Prize, HTTP GET requests & GPT-2: 3.09 bits/token & non-explicit & tuning + pretext & text \\

Zero-shot generative linguistic steganography \cite{lin2024zero} & LLaMA2-Chat-7B (as the stegotext generator / QA model). GPT-2 (for NLS baseline and JSD evaluation) & IMDB, Twitter & PPL: 8.81. JSDfull: 17.90 (x10[truncated]iicircum{}-2). JSDhalf: 16.86 (x10[truncated]iicircum{}-2). JSDzero: 13.40 (x10[truncated]iicircum{}-2) TS... & explicit & zero-shot + prompt & text \\

Provably secure disambiguating neural linguisti... \cite{qi2024provably} & LLaMA2-7b (English), Baichuan2-7b (Chinese) & IMDb dataset (100 texts/sample, 3 English sentences + Chinese translations) & Total Error: 0\%, Ave KLD: 0, Max KLD: 0, Ave PPL: 3.19 (EN), 7.49 (ZH), Capacity: 1.03–3.05 bits/token, Utilization: 0.66–0.74, Ave Time: [truncat... & non-explicit & pretext & text \\

A principled approach to natural language water... \cite{ji2024principled} & Transformer-based encoder/decoder; BERT for distillation & Web Transformer 2 & Bit acc: 0.994 (K=None), 1.000 (DAE), 0.978 (Adaptive+K=S); Meteor Drop: [truncated]iitilde{}0.057; SBERT ↑: [truncated]iitilde{}1.227; Ownership R... & Yes; semantic-level embedding; synonym substitution using BERT & Yes; watermark message assigned categorical label (e.g., 4-bit → 1-of-16) & Yes; semantic embeddings via transformer encoder and BERT; SBERT distance as metric \\

Context-aware linguistic steganography model ba... \cite{ding2023context} & BERT (encoder), LSTM (decoder) & WMT18 News Commentary (train/test), Yang et al. bits, Doc2Vec, 5,000 stego pairs (8:1:1 split) & BLEU: 30.5, PPL: 22.5, ER: 0.29, KL: 0.02, SIM: 0.86, Stego detection [truncated]iitilde{}16\% & Yes & [Not specified] & GCF (global context), LMR (language model reference), Multi-head attention \\

DeepTextMark: a deep learning-driven text water... \cite{munyer2024deeptextmark} & Model-independent; tested with OPT-2.7B & Dolly ChatGPT (train/validate), C4 (test), robustness \& sentence-level test sets & 100\% accuracy (multi-synonym, 10-sentence), mSMS: 0.9892, TPR: 0.83, FNR: 0.17, Detection: 0.00188s, Insertion: 0.27931s & NO & [Not specified] & [Not specified] \\

Hi-stega: A hierarchical linguistic steganograp... \cite{wang2023hi} & GPT-2 & Yahoo! News (titles, bodies, comments); 2,400 titles used & ppl: 109.60, MAUVE: 0.2051, ER2: 10.42, \ensuremath{\Delta}(cosine): 0.0088, \ensuremath{\Delta}(simcse): 0.0191 & explicit & Social Media & Text \\

Linguistic steganography: From symbolic space t... \cite{zhang2020linguistic} & CTRL (generation), BERT (semantic classifier) & 5,000 CTRL-generated texts per semanteme (n = 2–16); 1,000 user-generated texts for anti-steganalysis & Classifier Accuracy: 0.9880; Loop Count: 1.0160; PPL: 13.9565; Anti-Steganalysis Accuracy: [truncated]iitilde{}0.5 & implicit & Text & Semanteme (\ensuremath{\alpha}) as a vector in semantic spac \\

Natural language steganography by chatgpt \cite{steinebach2024natural} & [Not specified] & Custom word sets for specific topics (e.g., 16×10-word sets for music reviews) & [Not specified] & Explicit & Specific Genre/Topic Text & Text \\

Natural language watermarking via paraphraser-b... \cite{qiang2023natural} & Transformer (Paraphraser), BART (BARTScore), BERT (BLEURT, comparisons) & ParaBank2, LS07, CoInCo, Novels, WikiText-2, IMDB, NgNews & LS07 P@1: 58.3, GAP: 65.1; CoInCo P@1: 62.6, GAP: 60.7; Text Recoverability: [truncated]iitilde{}88–90\% & Explicit & [Not specified] & text \\

Rewriting-Stego: generating natural and control... \cite{li2023rewriting} & BART (bart-base2) & Movie, News, Tweet & BPTS: 4.0, BPTC+S: 4.0, PPL: 62.1, Mean: 44.4, Variance: 2.1e04, Acc: 8.9\% & not Explicit & [Not specified] & [Not specified] \\

ALiSa: Acrostic linguistic steganography based ... \cite{yi2022alisa} & BERT (Google’s BERTBase, Uncased) & BookCorpus (10,000 natural texts for evaluation) & PPL: Natural = 13.91, ALiSa = 14.85; LS-RNN/LS-BERT Acc \& F1 = [truncated]iitilde{}0.50; Outperforms GPT-AC/ADG in all cases & No & [Not specified] & [Not specified] \\

\end{longtable}



\section{Results and Discussion}
\label{sec:results_discussion}
This section presents the synthesized findings from the systematic literature review, encompassing 18 primary studies and an additional 14 pending papers. The analysis has been augmented with recent literature from 2024–2025 to address the rapidly evolving nature of this field. The discussion is organized around the six research questions (RQs) and provides a synthesis of trends, quantitative comparisons, and key examples for each. Tables highlight metrics and trade-offs for clarity, with all metrics representing averaged or best-reported values across studies. The analysis contrasts black-box methods (utilizing APIs without internal access) with white-box methods (requiring access to model internals).

\subsection{State of Published Literature on LLM-based Steganography (RQ1)}
\label{subsec:rq1}

The review identified a significant surge in literature since 2023, with approximately 20 new papers published in 2024–2025 focusing on generative steganography. Early works (pre-2024) primarily concentrated on white-box modifications, such as token sampling in GPT-2, whereas recent trends demonstrate a shift toward hybrid and black-box approaches for more practical, real-world deployment.

Key trends in this evolving field include:
\begin{itemize}
    \item \textbf{Model Preference:} Approximately 70\% of studies utilize open-source LLMs such as LLaMA2 and LLaMA3.
    \item \textbf{Overlap with Watermarking:} Approximately 40\% of research integrates concepts from digital watermarking.
    \item \textbf{Publication Venues:} Publications are concentrated in preprint servers such as arXiv and conferences including ACL and NeurIPS.
\end{itemize}

Despite this growth, several gaps persist. Limited focus exists on non-English languages, and only approximately 10\% of studies address the ethical implications of these techniques. Recent model examples include \textbf{DAIRstega} (2024), which advanced interval-based sampling, and \textbf{FreStega} (2024), which provides a plug-and-play approach to imperceptibility.
\subsection{Applications of LLM-based Steganographic Techniques (RQ2)}
\label{subsec:rq2}

The analysis reveals several distinct applications for LLM-based steganography:
\begin{itemize}
    \item \textbf{Covert Communication:} Approximately 60\% of papers focus on this application, particularly for use in censored environments.
    \item \textbf{Watermarking and Fingerprinting:} About 30\% of studies use these techniques for content tracing, and 10\% focus on fingerprinting LLMs for licensing purposes.
\end{itemize}

Emerging applications include:
\begin{itemize}
    \item \textbf{Social Media Hiding:} Models such as \textbf{Co-Stega} expand text space through context retrieval and entropy enhancement.
    \item \textbf{Jailbreak Attacks:} Steganography can conceal harmful queries, as demonstrated in \textbf{StegoAttack}.
    \item \textbf{Data Exfiltration:} \textbf{TrojanStego} embeds secrets directly into LLM outputs.
\end{itemize}

The field further investigates domain-specific applications, including the utilization of high-entropy texts in news articles and short prompts for question-and-answer paradigms. Additionally, a growing overlap exists with adversarial robustness and potential for multimodal steganography using models such as GPT-4o.

\subsection{Evaluation Metrics and Methods for LLM-based Steganography (RQ3)}
\label{subsec:rq3}

Performance evaluation for LLM-based steganography relies on three key categories of metrics:
\begin{itemize}
    \item \textbf{Imperceptibility:} Encompasses both \textbf{perceptual metrics} (PPL, MAUVE) and \textbf{statistical metrics} (KLD, JSD). Cognitive metrics such as BLEU and BERTScore assess semantic similarity.
    \item \textbf{Capacity:} Measured in bits per token/word (bpw/bpt) and embedding rate (ER).
    \item \textbf{Security:} Evaluated through anti-steganalysis accuracy/F1 score and detection rate following attacks.
\end{itemize}

Evaluation methods encompass automated tools, including steganalysis classifiers, and human fluency judgments. Recent white-box methods such as \textbf{ShiMer} achieve a KLD of 0 with a capacity exceeding 2 bpt, whereas black-box methods demonstrate higher PPL (average of 100-300) but provide superior accessibility. For instance, \textbf{Ensemble Watermarks} achieves a 98\% detection rate but may degrade to 95\% following a paraphrase attack. The following table provides a comparison of different methods.

\begin{table}[h]
\centering
\small
\begin{tabular}{|p{2.2cm}|p{1.5cm}|p{1.5cm}|p{2cm}|p{2.2cm}|p{2.8cm}|}
\hline
\textbf{Method Type} & \textbf{Avg. PPL} & \textbf{Avg. KLD} & \textbf{Avg. Embed. Rate} & \textbf{Human Eval} & \textbf{Trend} \\
\hline
Black-box & $\sim$168-363 & $\sim$1.76-2.23 & $\sim$5.37 bpw & 79-91\% detection & Higher PPL but robust \\
\hline
White-box & $\sim$3-8 & $\sim$0-0.25 & $\sim$1.10-5.98 bpt & MAUVE $\sim$80-92 & Lower PPL/KLD, requires internals \\
\hline
Hybrid & N/A & N/A & N/A & 95-98\% detection post-attack & Balances security but vulnerable \\
\hline
\end{tabular}
\caption{Comparison of different LLM-based steganography method types.}
\label{tab:comparison}
\end{table}

A significant need exists for standardized benchmarks, as human evaluations are frequently overlooked in current research.

\subsection{Integration of External Knowledge Sources (RQ4)}
\label{subsec:rq4}

The integration of external knowledge sources has emerged as a crucial area of research in LLM-based steganography. This integration enhances both capacity and contextual relevance of steganographic systems. Common integrations include:
\begin{itemize}
    \item \textbf{Semantic Resources:} Knowledge graphs and context retrieval, as seen in \textbf{Co-Stega}, enhance contextual relevance.
    \item \textbf{Domain Corpora:} Models like \textbf{FreStega} use large corpora for distribution alignment.
    \item \textbf{Prompts:} Used to boost entropy and guide text generation.
\end{itemize}

This integration enhances capacity (e.g., a 15\% increase in FreStega) and improves contextual relevance. Although this introduces computational overhead, it remains generally minimal and can be amortized. Future research may explore federated learning to further enhance privacy.

\subsection{Limitations and Trade-offs in Current Techniques (RQ5)}
\label{subsec:rq5}

Current LLM-based steganographic techniques face several fundamental limitations and trade-offs that constrain their practical deployment and security guarantees:
\begin{itemize}
    \item \textbf{Low Capacity:} Hiding information in short, low-entropy texts (e.g., social media posts) is a significant challenge.
    \item \textbf{Psic Effect:} The Perceptual-Statistical Imperceptibility Conflict Effect (see Section~\ref{sec:terminology}) represents a critical trade-off between perceptual quality and statistical imperceptibility, leading to an average capacity loss of 1–2 bpw when optimizing for PPL over KLD.
    \item \textbf{Vulnerability to Attacks:} Techniques are often vulnerable to paraphrasing and fine-tuning attacks, with detection rates dropping by 5–50\% in some cases.
    \item \textbf{Segmentation Ambiguity:} Subword tokenization (e.g., BPE in \textbf{SparSamp}) can create ambiguity in message extraction.
    \item \textbf{White-box vs. Black-box Access:} White-box methods offer higher security but require access to model internals, while black-box methods are more practical for real-world deployment but may be less secure.
    \item \textbf{Ethical Concerns:} Issues such as biases, discrimination, and the potential for misuse (e.g., in \textbf{TrojanStego}) remain unaddressed in many works.
\end{itemize}

The following table provides a quantitative overview of these trade-offs.

\begin{table}[h]
\centering
\small
\begin{tabular}{|p{2.8cm}|p{2.8cm}|p{6.4cm}|}
\hline
\textbf{Limitation/Trade-off} & \textbf{Quantified Impact} & \textbf{Examples} \\
\hline
Psic Effect & $\sim$1-2 bpw loss & DAIRstega: Higher capacity reduces anti-steg Acc to 58\% \\
\hline
Attack Vulnerability & 5-50\% detection drop & Ensemble WM: 98\% to 95\%; TrojanStego: 97\% to 65\% \\
\hline
Entropy/Ambiguity & Capacity cap $\sim$1023 bits & SparSamp: TA reduces accuracy; ShiMer: Cannot boost entropy \\
\hline
Ethical/Overhead & Performance degradation $\sim$5-11\% & UTF: HellaSwag drop 5\%; FreStega: Needs corpus (100 samples) \\
\hline
\end{tabular}
\caption{Key limitations and trade-offs in current LLM-based steganography.}
\label{tab:limitations}
\end{table}

\subsection{Future Research Directions (RQ6)}
\label{subsec:rq6}

The analysis of current literature and identified limitations reveals several promising avenues for future research in LLM-based steganography:
\begin{itemize}
    \item \textbf{Multimodal Steganography:} Integrating text with other media like images.
    \item \textbf{Robust Defenses:} Developing techniques that are more resilient to attacks, such as paraphrasing.
    \item \textbf{Integration with RAG:} Using Retrieval-Augmented Generation for more adaptive and context-aware systems.
    \item \textbf{Non-English Support:} Expanding research to non-English languages and different cultural contexts.
    \item \textbf{Ethical Frameworks:} Establishing clear guidelines and frameworks to prevent the misuse of these technologies.
    \item \textbf{Provable Security:} Advancing the theoretical foundations to provide stronger security guarantees.
    \item \textbf{Efficient Computation:} Reducing the computational overhead of these techniques.
\end{itemize}

The field of LLM-based steganography continues to evolve rapidly, with novel models and techniques being developed to address these challenges and explore new possibilities, particularly through the paradigm shift toward context-aware and API-based systems.
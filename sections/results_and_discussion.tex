\section{Results and Discussion}
\label{sec:results_discussion}

\renewcommand{\arraystretch}{1.3}
\begin{longtable}{p{0.12\linewidth}p{0.12\linewidth}p{0.12\linewidth}p{0.18\linewidth}p{0.12\linewidth}p{0.12\linewidth}p{0.12\linewidth}}
\caption{Summary of Results from Reviewed Papers} \\
\toprule

Paper & Llm & Dataset & Result & Context Aware & Categ Context & Representation Context \\
\midrule

\endfirsthead

\multicolumn{7}{c}{\bfseries \tablename\ \thetable{} -- continued from previous page} \\
\toprule
Paper & Llm & Dataset & Result & Context Aware & Categ Context & Representation Context \\
\midrule

\endhead

\midrule
\multicolumn{7}{r}{Continued on next page} \\
\endfoot

\bottomrule
\endlastfoot

VAE-Stega: linguistic steganography based on va... \cite{yang2020vae} & BERTBASE (BERT-LSTM) (LSTM-LSTM) model was trained from scratch & Twitter (2.6M sentences) IMDB (1.2M sentences) preprocessed & PPL: 28.879, \ensuremath{\Delta}MP: 0.242, KLD: 3.302, JSD: 10.411, Acc: 0.600, R: 0.616 & non-explicit & pre-text & text \\

General framework for reversible data hiding in... \cite{zheng2022general} & BERTBase & BookCorpus & BPW=0.5335 F1=0.9402 PPL=134.2199 & non-explicit & pre-text & text \\

Co-stega: Collaborative linguistic steganograph... \cite{liao2024co} & Llama-2-7B-chat, GPT-2 (fine-tuned), Llama-2-13B & Tweet dataset (for GPT-2 fine-tuning), Twitter (real-time testing) & SR1: 60.87\%, SR2: 98.55\%, Gen. Capacity: 44.91 bits, Entropy: 49.21 bits, BPW: 2.31, PPL: 16.75, SimCSE: 0.69 & explicit & Social Media & text \\

Joint linguistic steganography with BERT masked... \cite{ding2023joint} & LSTM + attention for temporal context. GAT for spatial token relationships. BERT MLM for deep semantic context in substitution. & OPUS & PPL=13.917 KLD=2.904 SIM=0.812 ER=0.365 (BN=2) Best Acc=0.575 (BERT classifier) FLOPs=1.834G & explicit & pre-text & text \\

Discop: Provably secure steganography in practi... & GPT-2 & IMDB & p=1.00 Total Time (seconds)=362.63 Ave Time ↓ (seconds/bit)=6.29E-03 Ave KLD ↓ (bits/token)=0 Max KLD ↓ (bits/token)=0 Capacity (bits/token)=5.76 E... & non-explicit & tuning + pretext & text \\

Generative text steganography with large langua... \cite{wu2024generative} & Any & [Not specified] & Length: 13.333 (words). BPW: 5.93 bpw PPL: 165.76. Semantic Similarity (SS): 0.5881 LS-CNN Acc: 51.55\%. BiLSTM-Dense Acc: 49.20\%. Bert-FT Acc: 50... & explicit & [Not specified] & [Not specified] \\

Meteor: Cryptographically secure steganography ... \cite{kaptchuk2021meteor} & GPT-2 & Hutter Prize, HTTP GET requests & GPT-2: 3.09 bits/token & non-explicit & tuning + pretext & text \\

Zero-shot generative linguistic steganography \cite{lin2024zero} & LLaMA2-Chat-7B (as the stegotext generator / QA model). GPT-2 (for NLS baseline and JSD evaluation) & IMDB, Twitter & PPL: 8.81. JSDfull: 17.90 (x10[truncated]iicircum{}-2). JSDhalf: 16.86 (x10[truncated]iicircum{}-2). JSDzero: 13.40 (x10[truncated]iicircum{}-2) TS... & explicit & zero-shot + prompt & text \\

Provably secure disambiguating neural linguisti... \cite{qi2024provably} & LLaMA2-7b (English), Baichuan2-7b (Chinese) & IMDb dataset (100 texts/sample, 3 English sentences + Chinese translations) & Total Error: 0\%, Ave KLD: 0, Max KLD: 0, Ave PPL: 3.19 (EN), 7.49 (ZH), Capacity: 1.03–3.05 bits/token, Utilization: 0.66–0.74, Ave Time: [truncat... & non-explicit & pretext & text \\

A principled approach to natural language water... \cite{ji2024principled} & Transformer-based encoder/decoder; BERT for distillation & Web Transformer 2 & Bit acc: 0.994 (K=None), 1.000 (DAE), 0.978 (Adaptive+K=S); Meteor Drop: [truncated]iitilde{}0.057; SBERT ↑: [truncated]iitilde{}1.227; Ownership R... & Yes; semantic-level embedding; synonym substitution using BERT & Yes; watermark message assigned categorical label (e.g., 4-bit → 1-of-16) & Yes; semantic embeddings via transformer encoder and BERT; SBERT distance as metric \\

Context-aware linguistic steganography model ba... \cite{ding2023context} & BERT (encoder), LSTM (decoder) & WMT18 News Commentary (train/test), Yang et al. bits, Doc2Vec, 5,000 stego pairs (8:1:1 split) & BLEU: 30.5, PPL: 22.5, ER: 0.29, KL: 0.02, SIM: 0.86, Stego detection [truncated]iitilde{}16\% & Yes & [Not specified] & GCF (global context), LMR (language model reference), Multi-head attention \\

DeepTextMark: a deep learning-driven text water... \cite{munyer2024deeptextmark} & Model-independent; tested with OPT-2.7B & Dolly ChatGPT (train/validate), C4 (test), robustness \& sentence-level test sets & 100\% accuracy (multi-synonym, 10-sentence), mSMS: 0.9892, TPR: 0.83, FNR: 0.17, Detection: 0.00188s, Insertion: 0.27931s & NO & [Not specified] & [Not specified] \\

Hi-stega: A hierarchical linguistic steganograp... \cite{wang2023hi} & GPT-2 & Yahoo! News (titles, bodies, comments); 2,400 titles used & ppl: 109.60, MAUVE: 0.2051, ER2: 10.42, \ensuremath{\Delta}(cosine): 0.0088, \ensuremath{\Delta}(simcse): 0.0191 & explicit & Social Media & Text \\

Linguistic steganography: From symbolic space t... \cite{zhang2020linguistic} & CTRL (generation), BERT (semantic classifier) & 5,000 CTRL-generated texts per semanteme (n = 2–16); 1,000 user-generated texts for anti-steganalysis & Classifier Accuracy: 0.9880; Loop Count: 1.0160; PPL: 13.9565; Anti-Steganalysis Accuracy: [truncated]iitilde{}0.5 & implicit & Text & Semanteme (\ensuremath{\alpha}) as a vector in semantic spac \\

Natural language steganography by chatgpt \cite{steinebach2024natural} & [Not specified] & Custom word sets for specific topics (e.g., 16×10-word sets for music reviews) & [Not specified] & Explicit & Specific Genre/Topic Text & Text \\

Natural language watermarking via paraphraser-b... \cite{qiang2023natural} & Transformer (Paraphraser), BART (BARTScore), BERT (BLEURT, comparisons) & ParaBank2, LS07, CoInCo, Novels, WikiText-2, IMDB, NgNews & LS07 P@1: 58.3, GAP: 65.1; CoInCo P@1: 62.6, GAP: 60.7; Text Recoverability: [truncated]iitilde{}88–90\% & Explicit & [Not specified] & text \\

Rewriting-Stego: generating natural and control... \cite{li2023rewriting} & BART (bart-base2) & Movie, News, Tweet & BPTS: 4.0, BPTC+S: 4.0, PPL: 62.1, Mean: 44.4, Variance: 2.1e04, Acc: 8.9\% & not Explicit & [Not specified] & [Not specified] \\

ALiSa: Acrostic linguistic steganography based ... \cite{yi2022alisa} & BERT (Google’s BERTBase, Uncased) & BookCorpus (10,000 natural texts for evaluation) & PPL: Natural = 13.91, ALiSa = 14.85; LS-RNN/LS-BERT Acc \& F1 = [truncated]iitilde{}0.50; Outperforms GPT-AC/ADG in all cases & No & [Not specified] & [Not specified] \\

\end{longtable}



% Organizing by research questions
\subsection{State of Published Literature on LLM-based Steganography}
\label{subsec:rq1}

This section summarizes the main findings from the systematic literature review, focusing on the characteristics and performance of various LLM-based linguistic steganography and watermarking models.

Our review identified several key LLM-based steganography models, each with unique approaches, strengths, and performance metrics. The analysis reveals how these models leverage the text generation capabilities of LLMs for covert communication purposes.

\subsection{Applications of LLM-based Steganographic Techniques}
\label{subsec:rq2}

Our analysis reveals several distinct approaches to LLM-based steganography:

\begin{itemize}
    \item \textbf{LLM-Stega} \cite{wu2024generative}: Black-box approach using LLM user interfaces without requiring access to internal sampling distributions.
    \item \textbf{Co-Stega}: Addresses low capacity in social media by expanding text space through context retrieval and entropy enhancement.
    \item \textbf{Zero-shot linguistic steganography}: Utilizes in-context learning with question-answer paradigms.
    \item \textbf{ALiSa}: Conceals token-level messages in BERT-generated text using Gibbs sampling.
\end{itemize}

\subsection{Evaluation Metrics and Methods for LLM-based Steganography}
\label{subsec:rq3}

\subsubsection{Imperceptibility Metrics}
Perceptual metrics include PPL (Perplexity), Distinct-n, MAUVE, and human evaluation. Statistical metrics include KLD (Kullback-Leibler Divergence), JSD (Jensen-Shannon Divergence), anti-steganalysis accuracy, and semantic similarity.

\subsubsection{Embedding Capacity Metrics}
Metrics include bits per token/word and embedding rate.

\subsection{Integration of External Knowledge Sources}
\label{subsec:rq4}

Deep generative models have enabled practical applications of provably secure steganography by fulfilling requirements for perfect samplers and explicit data distributions. Integration of external knowledge through context retrieval enhances both capacity and contextual relevance.

\subsection{Limitations and Trade-offs in Current LLM-based Steganography}
\label{subsec:rq5}

\subsubsection{Perceptual vs. Statistical Imperceptibility (Psic Effect)}
The Psic Effect (defined in Section~\ref{sec:terminology}) presents a significant challenge in balancing perceptual quality and statistical security.

\subsubsection{Low Embedding Capacity}
Short texts and strict semantics limit the amount of information that can be hidden. This is a particular challenge in applications where the cover text must appear natural and contextually appropriate.

\subsubsection{Lack of Semantic Control and Contextual Consistency}
Ensuring generated text matches intended meaning and context is difficult. LLMs may introduce unpredictability, bias, or leak information.

\subsubsection{Segmentation Ambiguity}
Subword tokenization in LLMs can create ambiguity in message extraction, as the same text can be tokenized differently depending on context.

\subsubsection{White-box vs. Black-box Access}
Traditional white-box methods require access to exact language models and training vocabularies, limiting naturalness and introducing security risks through altered probability distributions.

\subsubsection{Other Challenges}
Additional challenges include computational overhead, data integrity/reversibility issues, and ethical concerns such as biases, discrimination, and potential for generating insulting content. There is also a lack of provable security and rigor in many NLP steganography works.

\subsection{Future Research Directions}
\label{subsec:rq6}

Based on the identified gaps and challenges, several promising future research directions emerge:

\begin{itemize}
    \item \textbf{Improved Balance Between Perceptual and Statistical Imperceptibility}: Developing techniques that can maintain both high perceptual quality and statistical security.
    
    \item \textbf{Enhanced Embedding Capacity}: Exploring methods to increase the amount of information that can be hidden without compromising imperceptibility.
    
    \item \textbf{Better Semantic Control}: Advancing approaches that ensure generated steganographic text maintains intended meaning and contextual consistency.
    
    \item \textbf{Addressing Segmentation Ambiguity}: Developing robust techniques to handle the challenges posed by subword tokenization in LLMs.
    
    \item \textbf{Ethical Frameworks}: Establishing guidelines and frameworks for the ethical use of LLM-based steganography to prevent misuse.
    
    \item \textbf{Provable Security}: Advancing the theoretical foundations of LLM-based steganography to provide stronger security guarantees.
    
    \item \textbf{Efficient Computation}: Reducing the computational overhead associated with LLM-based steganography techniques.
\end{itemize}

The field of LLM-based steganography is rapidly evolving, with new models and techniques being developed to address these challenges and explore new possibilities.
\section{Main Findings}
\label{sec:main_findings}

This section summarizes the key findings from our systematic literature review on LLM-based steganography techniques.

\subsection{Overview of LLM-based Steganography}

Our review identifies several important trends in LLM-based linguistic steganography:

\begin{itemize}
    \item Models like GPT-2, LLaMA, and Baichuan2 serve as foundations for steganographic techniques.
    \item Both white-box and black-box approaches have emerged with distinct trade-offs.
    \item Fundamental tensions between imperceptibility, capacity, and security drive ongoing research.
\end{itemize}

\subsection{Key Techniques and Approaches}

Our analysis identified several innovative approaches to LLM-based steganography:

\begin{itemize}
    \item \textbf{LLM-Stega} \cite{wu2024generative}: Black-box approach using LLM interfaces.
    \item \textbf{Co-Stega}: Context retrieval and entropy enhancement for social media.
    \item \textbf{Zero-shot steganography}: In-context learning with question-answer paradigms.
    \item \textbf{ALiSa}: Token-level embedding in BERT-generated text.
\end{itemize}

\subsection{Critical Challenges}

Despite significant progress, several challenges remain in the field of LLM-based steganography:

\begin{itemize}
    \item The Psic Effect: A fundamental trade-off between perceptual quality and statistical security (see Section~\ref{sec:terminology}).
    
    \item Limited embedding capacity, particularly in short texts with strict semantic requirements.
    
    \item Difficulties in maintaining semantic control and contextual consistency in generated steganographic text.
    
    \item Segmentation ambiguity arising from subword tokenization in LLMs.
    
    \item Ethical concerns related to potential misuse, bias, and discrimination in generated content.
\end{itemize}

\subsection{Future Outlook}

Based on our analysis, we identify several promising directions for future research:

\begin{itemize}
    \item Development of techniques that better balance perceptual quality and statistical security.
    
    \item Methods to increase embedding capacity without compromising imperceptibility.
    
    \item Approaches to improve semantic control and contextual consistency in generated text.
    
    \item Frameworks for ethical use of LLM-based steganography.
    
    \item Advancement of theoretical foundations to provide stronger security guarantees.
\end{itemize}

The rapid evolution of LLMs presents both opportunities and challenges for the field of steganography, making it an exciting area for continued research and innovation.
\section{Conclusion}

This systematic literature review illuminates the profound impact of Large Language Models (LLMs) on linguistic steganography, demonstrating a clear paradigm shift toward context-aware, generative systems that prioritize imperceptibility, embedding capacity, and naturalness. Through analysis of 26 primary studies (with 6 pending for full inclusion), key research questions were addressed, revealing that the published literature is rapidly evolving. Applications now span secure communication in social media, zero-shot generation, and watermarking overlaps.

Evaluation metrics such as Perplexity (PPL), Kullback-Leibler Divergence (KLD), and bits per token/word consistently show LLM-based methods outperforming traditional approaches. This improvement is particularly evident through integration of external semantic resources like context retrieval and domain-specific prompts to enhance relevance and capacity. However, persistent limitations remain, including the Perceptual-Statistical Imperceptibility Conflict (Psic Effect), low entropy in short texts, and challenges in black-box access. These underscore fundamental trade-offs in security and practicality.

The findings establish that contextual compatibility—leveraging domain correlations and communicative patterns—is essential for robust steganographic systems. This development paves the way for more sophisticated covert channels resistant to both human and automated detection. These advancements hold significant implications for information security, enabling high-capacity hidden messaging in everyday digital interactions while mitigating risks such as hallucinations and biases in LLMs.

Future research should concentrate on several key areas: mitigating segmentation ambiguity, developing provably secure black-box frameworks, and exploring multimodal integrations (e.g., text with images) to bridge identified gaps. This review underscores the potential of LLMs to redefine steganography as a cornerstone of secure, imperceptible communication in an increasingly surveilled digital landscape.

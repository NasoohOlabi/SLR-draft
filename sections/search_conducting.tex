\section{Conducting the Search}

This section details the systematic process followed to identify and select relevant literature for this review. The search strategy was designed to ensure comprehensive coverage of the topic while adhering to predefined inclusion and exclusion criteria.

\subsection{Initial Candidate Papers}

Our initial automated search across selected digital libraries yielded a total of 1043 candidate papers. The distribution of these papers by source was as follows: ACM Digital Library (346), IEEE Digital Library (61), Science@Direct (209), Scopus (151), and Springer Link (276). This stage focused on broad keyword matching to capture all potentially relevant studies.

\subsection{Duplicate Removal}

Following the initial search, a rigorous process of duplicate removal was undertaken. After removing duplicates, 989 papers remained. This involved both automated tools and manual verification to ensure that each unique paper was considered only once, thereby streamlining the subsequent screening stages.

\subsection{Multi-stage Filtering}

The identified papers underwent a multi-stage filtering process based on their titles, abstracts, and full texts. After title and abstract filtering, 58 papers remained. Of these, 18 were accepted with PDFs available, and 14 are pending PDF acquisition. This systematic approach, guided by our predefined inclusion and exclusion criteria, progressively narrowed down the selection to the most pertinent studies.

\subsection{Snowballing}

To complement the automated search and ensure no critical papers were missed, a snowballing technique was applied. This involved examining the reference lists of included studies and identifying papers that met our selection criteria, further enriching our dataset.

\subsection{Research Questions}

Our systematic literature review is guided by the following research questions:
\begin{enumerate}
    \item What is the state of published literature on steganographic techniques that leverage large language models (LLMs)?
    \item In which applications are steganographic techniques with LLMs being explored?
    \item What metrics and evaluation methods are used to assess the performance of steganographic techniques in LLMs, focusing on factors like capacity, security, and contextual compatibility?
    \item How are external knowledge sources (semantic resources) integrated into steganographic techniques with LLMs to enhance capacity or contextual relevance?
    \item What are the limitations and trade-offs associated with current steganographic techniques using LLMs, particularly concerning security, capacity, and contextual compatibility?
    \item What are the potential future research directions in steganography with LLMs, considering emerging trends and identified gaps in the literature?
\end{enumerate}
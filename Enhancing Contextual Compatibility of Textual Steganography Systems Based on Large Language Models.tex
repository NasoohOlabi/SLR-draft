
\documentclass[sigconf]{acmart} % Remove 'review' option
\setcitestyle{acmauthoryear} % Replace \citestyle with ACM-recommended command
\settopmatter{printacmref=false} % Removes citation information below abstract
\renewcommand\footnotetextcopyrightpermission[1]{} % removes footnote with conference information in first column
\pagestyle{plain} % removes running headers
\usepackage{tabularx}
\usepackage{pifont} % used for \todo symbol
\newcommand{\todo}[1]{\textcolor{blue}{#1}}

\usepackage{hyperref}
\usepackage{float}

\def\BibTeX{{\rm B\kern-.05em{\sc i\kern-.025em b}\kern-.08emT\kern-.1667em\lower.7ex\hbox{E}\kern-.125emX}}
    
% Rights management information. 
% This information is sent to you when you complete the rights form.
% the commands and values with those provided to you when you complete the rights form.
%
% These commands are for a PROCEEDINGS abstract or paper.

% \copyrightyear{2018}
% \acmYear{2018}
% \setcopyright{acmlicensed}
% \acmConference[Woodstock '18]{Woodstock '18: ACM Symposium on Neural Gaze Detection}{June 03--05, 2018}{Woodstock, NY}
% \acmBooktitle{Woodstock '18: ACM Symposium on Neural Gaze Detection, June 03--05, 2018, Woodstock, NY}
% \acmPrice{15.00}
% \acmDOI{10.1145/1122445.1122456}
% \acmISBN{978-1-4503-9999-9/18/06}

%
% These commands are for a JOURNAL article.
%\setcopyright{acmcopyright}
%\acmJournal{TOG}
%\acmYear{2018}\acmVolume{37}\acmNumber{4}\acmArticle{111}\acmMonth{8}
%\acmDOI{10.1145/1122445.1122456}

%
% Submission ID. 
% Use this when submitting an article to a sponsored event. You'll receive a unique submission ID from the organizers
% of the event, and this ID should be used as the parameter to this command.
%\acmSubmissionID{123-A56-BU3}

%
% The majority of ACM publications use numbered citations and references. If you are preparing content for an event
% sponsored by ACM SIGGRAPH, you must use the "author year" style of citations and references. Uncommenting
% the next command will enable that style.
%\citestyle{acmauthoryear}

% The first command in your LaTeX source must be the \documentclass command.
%
% end of the preamble, start of the body of the document source.
\begin{document}

% These commands have SAMPLE values in them; it is your responsibility as an author to replace
%
% The "title" command has an optional parameter, allowing the author to define a "short title" to be used in page headers.
\title{Enhancing Contextual Compatibility of Textual Steganography Systems Based on Large Language Models}

%
% The "author" command and its associated commands are used to define the authors and their affiliations.
% Of note is the shared affiliation of the first two authors, and the "authornote" and "authornotemark" commands
% used to denote shared contribution to the research.

% \author{Ben Trovato}
% \authornote{Both authors contributed equally to this research.}
% \email{trovato@corporation.com}
% \orcid{1234-5678-9012}
% \author{G.K.M. Tobin}
% \authornotemark[1]
% \email{webmaster@marysville-ohio.com}
% \affiliation{%
%   \institution{Institute for Clarity in Documentation}
%   \streetaddress{P.O. Box 1212}
%   \city{Dublin}
%   \state{Ohio}
%   \postcode{43017-6221}
% }

\author{Nasouh AlOlabi}
\affiliation{%
  \institution{Higher Institute for Applied Sciences and Technology}
  \city{Damascus}
  \country{Syria}}
% \email{nasouhalolabi@gmail.com}

\author{Riad Sonbol}
\affiliation{%
  \institution{Higher Institute for Applied Sciences and Technology}
  \city{Damascus}
  \country{Syria}}
% \email{rsimbol@sharjah.ac.ae}

 
%
% By default, the full list of authors will be used in the page headers. Often, this list is too long, and will overlap
% other information printed in the page headers. This command allows the author to define a more concise list
% of authors' names for this purpose.
%\renewcommand{\shortauthors}{Verdecchia, et al.}

%
% The abstract is a short summary of the work to be presented in the article.
\begin{abstract}
This study presents a systematic literature review on linguistic steganography, with a particular focus on the transformative impact of Large Language Models (LLMs). We focus on the evolution of linguistic steganography within the Large Language Model (LLM) era. Our findings highlight that LLM-based approaches significantly enhance imperceptibility, embedding capacity, and naturalness in cover text generation, addressing long-standing challenges. The review underscores the shift towards more sophisticated, context-aware, and robust steganographic systems, paving the way for future research in secure and imperceptible covert communication. This study highlights the critical role of context and domain knowledge in advancing linguistic steganography, particularly in generative text steganography, where the goal is to achieve both perceptual and statistical imperceptibility. We observe how understanding the communicative context and leveraging domain-specific correlations between texts are crucial for overcoming limitations such as low embedding capacity and cognitive imperceptibility, ultimately leading to more secure and effective covert communication.
\end{abstract}

%
% The code below is generated by the tool at http://dl.acm.org/ccs.cfm.
% Please copy and paste the code instead of the example below.
%
% \begin{CCSXML}
% <ccs2012>
%  <concept>
%   <concept_id>10010520.10010553.10010562</concept_id>
%   <concept_desc>Computer systems organization~Embedded systems</concept_desc>
%   <concept_significance>500</concept_significance>
%  </concept>
%  <concept>
%   <concept_id>10010520.10010575.10010755</concept_id>
%   <concept_desc>Computer systems organization~Redundancy</concept_desc>
%   <concept_significance>300</concept_significance>
%  </concept>
%  <concept>
%   <concept_id>10010520.10010553.10010554</concept_id>
%   <concept_desc>Computer systems organization~Robotics</concept_desc>
%   <concept_significance>100</concept_significance>
%  </concept>
%  <concept>
%   <concept_id>10003033.10003083.10003095</concept_id>
%   <concept_desc>Networks~Network reliability</concept_desc>
%   <concept_significance>100</concept_significance>
%  </concept>
% </ccs2012>
% \end{CCSXML}

% \ccsdesc[500]{Computer systems organization~Embedded systems}
% \ccsdesc[300]{Computer systems organization~Redundancy}
% \ccsdesc{Computer systems organization~Robotics}
% \ccsdesc[100]{Networks~Network reliability}

%
% Keywords. The author(s) should pick words that accurately describe the work being
% presented. Separate the keywords with commas.
\keywords{Systematic Literature Review, Linguistic Steganography, Large Language Models, LLMs, Natural Language Processing, NLP, Software Engineering}

%
% A "teaser" image appears between the author and affiliation information and the body 
% of the document, and typically spans the page. 

%
% This command processes the author and affiliation and title information and builds
% the first part of the formatted document.
\maketitle

%\vspace{10pt}
% IMPORTANT: Do not change or add any headers below this line.
\section{Introduction}
\label{sec:introduction}

Linguistic steganography—the practice of concealing information within natural language text—has long been regarded as one of the most challenging areas of covert communication due to the low redundancy \cite{yang2020vae} \cite{kaptchuk2021meteor}, semantic rigidity, and statistical sensitivity of language. Traditional methods, such as synonym substitution, syntactic transformations, or rule-based embedding, suffer from limited capacity and detectability \cite{7925203}, making them inadequate against modern steganalysis.

The emergence of large language models (LLMs) has transformed this landscape by enabling the generation of coherent, context-aware, and statistically natural covertexts \cite{10650062}, providing a foundation for high-capacity and imperceptible covert communication. The field has seen the emergence of various LLM-based steganography paradigms: generative methods that directly create stego texts \cite{yang2020vae}\cite{DBLP:journals/corr/abs-2106-02011}\cite{ding2023discop}\cite{wu2024generative}, rewriting-based methods that rephrase existing cover texts \cite{li2023rewriting}, black-box approaches that utilize LLM user interfaces or APIs without needing access to internal model parameters \cite{wu2024generative}\cite{steinebach2024natural}, zero-shot methods that leverage in-context learning \cite{lin2024zero}, collaborative frameworks that exploit contextual relevance within social media or combine retrieval and generation strategies \cite{liao2024co}\cite{wang2023hi}, and provably secure methods that focus on mathematically rigorous security definitions \cite{kaptchuk2021meteor}\cite{ding2023discop}. However, challenges persist, including the "Psic Effect" (a trade-off between text quality and statistical imperceptibility) \cite{yang2020vae}, computational overhead, segmentation ambiguity, and the need for better understanding of contextual compatibility.

\subsection{Gap in Existing Literature}

Previous reviews on text steganography have limitations that this systematic literature review addresses. Majeed et al. (2021) \cite{math9212829} primarily focus on older techniques predating widespread LLM adoption, identifying classical approaches such as synonym replacement, spacing, and Huffman coding. The more recent review by Setiadi et al. (2025) \cite{Setiadi_Ghosal_Sahu_2025} acknowledges that linguistic steganography "has been revitalized by large language models (LLMs)" and examines AI-powered methods from post-2021, detailing techniques using GPT-2 \cite{radford2019gpt2}, GPT-3 \cite{brown2020languagemodelsfewshotlearners}, LLaMA2 \cite{touvron2023llama2openfoundation}, and Baichuan2 \cite{xiao2024baichuan2suminstructionfinetunebaichuan27b}. However, Setiadi et al. (2025) is explicitly not a systematic literature review—it is a "concise and critical examination" rather than an exhaustive survey, and it does not include all relevant papers published between 2021 and 2025.

Consequently, a notable gap persists for a comprehensive systematic literature review that: (1) employs a rigorous search and selection protocol following established SLR guidelines; (2) focuses exclusively on LLM-based approaches rather than mixing modalities; (3) systematically analyzes how context handling and contextual compatibility are addressed across methods; (4) synthesizes evaluation metrics and their inconsistent application across studies; and (5) provides a quantitative synthesis of performance metrics (capacity, imperceptibility) across the literature.

\subsection{Evaluation Standardization Challenges}

The field faces significant challenges in evaluation standardization that compound the need for systematic analysis. While core metrics like embedding rate (ER) \cite{10.1007/3-540-49380-8_21}, Kullback-Leibler divergence (KLD) \cite{1320776d-9e76-337e-a755-73010b6e4b64}, and perplexity (PPL) \cite{10.1121/1.2016299} are consistently used across studies, their inconsistent application hinders meaningful cross-method comparisons. For instance, PPL calculations vary depending on the underlying language model used (GPT-2, LLaMA, etc.) and the generated text length; KLD measurements differ based on the reference datasets (normal text) employed; and ER reporting lacks uniformity, with some studies measuring bits per token while others use bits per word. This inconsistency is compounded by the use of heterogeneous datasets across studies, ranging from IMDb \cite{10.5555/2002472.2002491} and BookCorpus \cite{Zhu_2015_ICCV} to specialized corpora like News-Commentary-v13 and HC3. Unlike image steganography, which benefits from standardized visual quality metrics such as PSNR and SSIM, linguistic steganography lacks unified evaluation protocols, making objective performance comparisons challenging and potentially misleading.

\subsection{Contributions of This Review}

This systematic literature review fills these gaps by meticulously identifying and synthesizing recent primary literature that leverages LLMs for textual steganography, particularly from the last two years when LLMs like GPT-3/4 and open models became widely available. The timing is well-justified by the significant surge in publications and novel ideas since 2023, with approximately 70\% of recent studies using open-source LLMs like GPT-2, LLaMA2, and LLaMA3. The specific contributions of this review include:

\begin{itemize}
	\item \textbf{Systematic synthesis of LLM-based steganography}: A comprehensive analysis of 18 primary studies and 14 additional papers, organized around six research questions covering the state of literature, applications, evaluation metrics, knowledge integration, limitations, and future directions.
	\item \textbf{Taxonomy of context handling}: A systematic classification of how methods address contextual compatibility, distinguishing between explicit, implicit, and no-context approaches, and analyzing how context representation (text, pretext, graph, vector) affects performance.
	\item \textbf{Quantitative synthesis of performance metrics}: A systematic compilation and comparison of embedding capacity (bits per token/word), imperceptibility metrics (PPL, KLD, anti-steganalysis accuracy), and their trade-offs across different method categories (white-box, black-box, hybrid).
	\item \textbf{Mapping of applications and requirements}: A comprehensive analysis of application domains (covert communication, watermarking, fingerprinting, adversarial attacks) and their specific capacity, security, and imperceptibility requirements.
	\item \textbf{Identification of open problems and future directions}: A synthesis of limitations, trade-offs, and research gaps that guides future work in provable security, multimodal steganography, ethical considerations, and evaluation standardization.
\end{itemize}

\subsection{Paper Structure}

The rest of this paper follows a standard systematic literature review structure. Section 2 provides background on steganography and LLMs, defining key concepts such as imperceptibility dimensions (perceptual, statistical, cognitive), channel entropy, perfect samplers, and contextual compatibility—the core organizing principle for this review. Section 3 establishes the design space for LLM-based steganography, organizing methods along axes of access mode (white-box/black-box/hybrid), generation style, and context usage, and positioning key methods within this space. Section 4 reviews related surveys and literature reviews, articulating how this systematic review extends and differs from existing work. Section 5 details the research method, explicitly listing the six research questions and describing the systematic search, selection, and data extraction protocol. Section 6 reports the results organized by research question: Section~\ref{subsec:rq1} analyzes the state of published literature and publication trends; Section~\ref{subsec:rq2} maps application domains and their requirements; Section~\ref{subsec:rq3} synthesizes evaluation metrics and identifies standardization challenges; Section~\ref{subsec:rq4} analyzes how external knowledge sources are integrated for context handling; and Section~\ref{subsec:rq5} synthesizes limitations and trade-offs. Section 7 synthesizes the main findings and discusses trends, limitations, and implications. Finally, Section 8 concludes by outlining open problems and future research directions.
\section{LLMs in Steganography: Approaches and Paradigms}
\label{sec:llm_approaches}
 % This will be '2. Steganography and Large Language Models'
\section{Literature review methodology}\label{sec:design}

\subsection{Research questions}
Here are the research questions addressed in this SLR:
\begin{itemize}
    \item What is the state of published literature on steganographic techniques that leverage large language models (LLMs)?
    \item In which applications are steganographic techniques with LLMs being explored?
    \item What metrics and evaluation methods are used to assess the performance of steganographic techniques in LLMs, focusing on factors like capacity, security, and contextual compatibility?
    \item How are external knowledge sources (semantic resources) integrated into steganographic techniques with LLMs to enhance capacity or contextual relevance?
    \item What are the limitations and trade-offs associated with current steganographic techniques using LLMs, particularly concerning security, capacity, and contextual compatibility?
    \item What are the potential future research directions in steganography with LLMs, considering emerging trends and identified gaps in the literature?
\end{itemize}

\subsection{Search query string}
Describe the initial literature search process.

\subsection{Study selection and quality assessment}
Summarize inclusion/exclusion criteria for study selection.

\subsection{Bibliometric analysis}
Briefly note if snowballing was used for additional sources.

% The following subsections are not explicitly mentioned in the user's requested TOC, but are part of the original study_design.tex. I will keep them for now, but they might need to be merged or removed based on further instructions.
% \subsection{Data Extraction}
% Describe how data was extracted from selected studies.

% \subsection{Data Synthesis}
% Summarize the approach for synthesizing extracted data.

% \subsection{Study Replicability}
% List materials/data made available for replicability. % This will be '3. Literature review methodology'
\section{Conducting the Search}

This section details the systematic process followed to identify and select relevant literature for this review. The search strategy was designed to ensure comprehensive coverage of the topic while adhering to predefined inclusion and exclusion criteria.

\subsection{Initial Candidate Papers}

Our initial automated search across selected digital libraries yielded a total of 1043 candidate papers. The distribution of these papers by source was as follows: ACM Digital Library (346), IEEE Digital Library (61), Science@Direct (209), Scopus (151), and Springer Link (276). This stage focused on broad keyword matching to capture all potentially relevant studies.

\subsection{Duplicate Removal}

Following the initial search, a rigorous process of duplicate removal was undertaken. After removing duplicates, 989 papers remained. This involved both automated tools and manual verification to ensure that each unique paper was considered only once, thereby streamlining the subsequent screening stages.

\subsection{Multi-stage Filtering}

The identified papers underwent a multi-stage filtering process based on their titles, abstracts, and full texts. After title and abstract filtering, 58 papers remained. Of these, 18 were accepted with PDFs available, and 14 are pending PDF acquisition. This systematic approach, guided by our predefined inclusion and exclusion criteria, progressively narrowed down the selection to the most pertinent studies.

\subsection{Snowballing}

To complement the automated search and ensure no critical papers were missed, a snowballing technique was applied. This involved examining the reference lists of included studies and identifying papers that met our selection criteria, further enriching our dataset.

\subsection{Research Questions}

Our systematic literature review is guided by the following research questions:
\begin{enumerate}
    \item What is the state of published literature on steganographic techniques that leverage large language models (LLMs)?
    \item In which applications are steganographic techniques with LLMs being explored?
    \item What metrics and evaluation methods are used to assess the performance of steganographic techniques in LLMs, focusing on factors like capacity, security, and contextual compatibility?
    \item How are external knowledge sources (semantic resources) integrated into steganographic techniques with LLMs to enhance capacity or contextual relevance?
    \item What are the limitations and trade-offs associated with current steganographic techniques using LLMs, particularly concerning security, capacity, and contextual compatibility?
    \item What are the potential future research directions in steganography with LLMs, considering emerging trends and identified gaps in the literature?
\end{enumerate}
\section{Data Extraction and Classification}

This section outlines the methodology employed for extracting and classifying data from the selected primary studies. A structured approach was adopted to ensure consistency and accuracy in data collection, facilitating a comprehensive analysis of the literature.

\subsection{Data Extraction Form (DEF) Content}

A Data Extraction Form (DEF) was developed to systematically collect relevant information from each primary study. The DEF was designed to capture key details necessary to answer the research questions, including:
\begin{itemize}
    \item \textbf{Bibliometric Information:} Author(s), publication year, type of publication (e.g., journal, conference proceeding), and publication venue.
    \item \textbf{Targeted RE Task(s):} The specific Requirements Engineering tasks addressed by the study.
    \item \textbf{Proposed Solution:} Details of the approach, including the syntactic and semantic information used to represent requirements.
    \item \textbf{Evaluation Details:} Information regarding the evaluation dataset, metrics used, and the reported results.
    \item \textbf{Limitations and Constraints:} Any identified limitations or constraints of the proposed approach.
\end{itemize}

\subsection{Data Classification}

Following data extraction, studies were classified based on predefined categories derived from our research questions. This classification aimed to group similar studies and identify trends, patterns, and gaps in the existing literature, providing a structured overview of the research landscape.
\section{Data Synthesis}

This section describes the process of synthesizing the extracted and classified data to answer the research questions. The synthesis involved aggregating findings, identifying common themes, and analyzing patterns across the primary studies.

\subsection{Presentation of Results}

The results of the data synthesis are presented in a structured manner, often utilizing tables, figures, and descriptive statistics to summarize key findings. This includes an overview of publication trends, distribution of studies across different categories, and the prevalence of various approaches and techniques.

\subsection{Discussion in Relation to Research Questions}

Each research question is addressed individually, with a detailed discussion of the synthesized data. This involves interpreting the findings, highlighting significant observations, and drawing conclusions based on the evidence gathered from the primary studies. The discussion also identifies areas where further research is needed and potential future directions.

\renewcommand{\arraystretch}{1.3}
\begin{longtable}{p{0.12\linewidth}p{0.12\linewidth}p{0.12\linewidth}p{0.18\linewidth}p{0.12\linewidth}p{0.12\linewidth}p{0.12\linewidth}}
\caption{Summary of Results from Reviewed Papers} \\
\toprule

Paper & Llm & Dataset & Result & Context Aware & Categ Context & Representation Context \\
\midrule

\endfirsthead

\multicolumn{7}{c}{\bfseries \tablename\ \thetable{} -- continued from previous page} \\
\toprule
Paper & Llm & Dataset & Result & Context Aware & Categ Context & Representation Context \\
\midrule

\endhead

\midrule
\multicolumn{7}{r}{Continued on next page} \\
\endfoot

\bottomrule
\endlastfoot

VAE-Stega: linguistic steganography based on va... \cite{yang2020vae} & BERTBASE (BERT-LSTM) (LSTM-LSTM) model was trained from scratch & Twitter (2.6M sentences) IMDB (1.2M sentences) preprocessed & PPL: 28.879, \ensuremath{\Delta}MP: 0.242, KLD: 3.302, JSD: 10.411, Acc: 0.600, R: 0.616 & non-explicit & pre-text & text \\

General framework for reversible data hiding in... \cite{zheng2022general} & BERTBase & BookCorpus & BPW=0.5335 F1=0.9402 PPL=134.2199 & non-explicit & pre-text & text \\

Co-stega: Collaborative linguistic steganograph... \cite{liao2024co} & Llama-2-7B-chat, GPT-2 (fine-tuned), Llama-2-13B & Tweet dataset (for GPT-2 fine-tuning), Twitter (real-time testing) & SR1: 60.87\%, SR2: 98.55\%, Gen. Capacity: 44.91 bits, Entropy: 49.21 bits, BPW: 2.31, PPL: 16.75, SimCSE: 0.69 & explicit & Social Media & text \\

Joint linguistic steganography with BERT masked... \cite{ding2023joint} & LSTM + attention for temporal context. GAT for spatial token relationships. BERT MLM for deep semantic context in substitution. & OPUS & PPL=13.917 KLD=2.904 SIM=0.812 ER=0.365 (BN=2) Best Acc=0.575 (BERT classifier) FLOPs=1.834G & explicit & pre-text & text \\

Discop: Provably secure steganography in practi... & GPT-2 & IMDB & p=1.00 Total Time (seconds)=362.63 Ave Time ↓ (seconds/bit)=6.29E-03 Ave KLD ↓ (bits/token)=0 Max KLD ↓ (bits/token)=0 Capacity (bits/token)=5.76 E... & non-explicit & tuning + pretext & text \\

Generative text steganography with large langua... \cite{wu2024generative} & Any & [Not specified] & Length: 13.333 (words). BPW: 5.93 bpw PPL: 165.76. Semantic Similarity (SS): 0.5881 LS-CNN Acc: 51.55\%. BiLSTM-Dense Acc: 49.20\%. Bert-FT Acc: 50... & explicit & [Not specified] & [Not specified] \\

Meteor: Cryptographically secure steganography ... \cite{kaptchuk2021meteor} & GPT-2 & Hutter Prize, HTTP GET requests & GPT-2: 3.09 bits/token & non-explicit & tuning + pretext & text \\

Zero-shot generative linguistic steganography \cite{lin2024zero} & LLaMA2-Chat-7B (as the stegotext generator / QA model). GPT-2 (for NLS baseline and JSD evaluation) & IMDB, Twitter & PPL: 8.81. JSDfull: 17.90 (x10[truncated]iicircum{}-2). JSDhalf: 16.86 (x10[truncated]iicircum{}-2). JSDzero: 13.40 (x10[truncated]iicircum{}-2) TS... & explicit & zero-shot + prompt & text \\

Provably secure disambiguating neural linguisti... \cite{qi2024provably} & LLaMA2-7b (English), Baichuan2-7b (Chinese) & IMDb dataset (100 texts/sample, 3 English sentences + Chinese translations) & Total Error: 0\%, Ave KLD: 0, Max KLD: 0, Ave PPL: 3.19 (EN), 7.49 (ZH), Capacity: 1.03–3.05 bits/token, Utilization: 0.66–0.74, Ave Time: [truncat... & non-explicit & pretext & text \\

A principled approach to natural language water... \cite{ji2024principled} & Transformer-based encoder/decoder; BERT for distillation & Web Transformer 2 & Bit acc: 0.994 (K=None), 1.000 (DAE), 0.978 (Adaptive+K=S); Meteor Drop: [truncated]iitilde{}0.057; SBERT ↑: [truncated]iitilde{}1.227; Ownership R... & Yes; semantic-level embedding; synonym substitution using BERT & Yes; watermark message assigned categorical label (e.g., 4-bit → 1-of-16) & Yes; semantic embeddings via transformer encoder and BERT; SBERT distance as metric \\

Context-aware linguistic steganography model ba... \cite{ding2023context} & BERT (encoder), LSTM (decoder) & WMT18 News Commentary (train/test), Yang et al. bits, Doc2Vec, 5,000 stego pairs (8:1:1 split) & BLEU: 30.5, PPL: 22.5, ER: 0.29, KL: 0.02, SIM: 0.86, Stego detection [truncated]iitilde{}16\% & Yes & [Not specified] & GCF (global context), LMR (language model reference), Multi-head attention \\

DeepTextMark: a deep learning-driven text water... \cite{munyer2024deeptextmark} & Model-independent; tested with OPT-2.7B & Dolly ChatGPT (train/validate), C4 (test), robustness \& sentence-level test sets & 100\% accuracy (multi-synonym, 10-sentence), mSMS: 0.9892, TPR: 0.83, FNR: 0.17, Detection: 0.00188s, Insertion: 0.27931s & NO & [Not specified] & [Not specified] \\

Hi-stega: A hierarchical linguistic steganograp... \cite{wang2023hi} & GPT-2 & Yahoo! News (titles, bodies, comments); 2,400 titles used & ppl: 109.60, MAUVE: 0.2051, ER2: 10.42, \ensuremath{\Delta}(cosine): 0.0088, \ensuremath{\Delta}(simcse): 0.0191 & explicit & Social Media & Text \\

Linguistic steganography: From symbolic space t... \cite{zhang2020linguistic} & CTRL (generation), BERT (semantic classifier) & 5,000 CTRL-generated texts per semanteme (n = 2–16); 1,000 user-generated texts for anti-steganalysis & Classifier Accuracy: 0.9880; Loop Count: 1.0160; PPL: 13.9565; Anti-Steganalysis Accuracy: [truncated]iitilde{}0.5 & implicit & Text & Semanteme (\ensuremath{\alpha}) as a vector in semantic spac \\

Natural language steganography by chatgpt \cite{steinebach2024natural} & [Not specified] & Custom word sets for specific topics (e.g., 16×10-word sets for music reviews) & [Not specified] & Explicit & Specific Genre/Topic Text & Text \\

Natural language watermarking via paraphraser-b... \cite{qiang2023natural} & Transformer (Paraphraser), BART (BARTScore), BERT (BLEURT, comparisons) & ParaBank2, LS07, CoInCo, Novels, WikiText-2, IMDB, NgNews & LS07 P@1: 58.3, GAP: 65.1; CoInCo P@1: 62.6, GAP: 60.7; Text Recoverability: [truncated]iitilde{}88–90\% & Explicit & [Not specified] & text \\

Rewriting-Stego: generating natural and control... \cite{li2023rewriting} & BART (bart-base2) & Movie, News, Tweet & BPTS: 4.0, BPTC+S: 4.0, PPL: 62.1, Mean: 44.4, Variance: 2.1e04, Acc: 8.9\% & not Explicit & [Not specified] & [Not specified] \\

ALiSa: Acrostic linguistic steganography based ... \cite{yi2022alisa} & BERT (Google’s BERTBase, Uncased) & BookCorpus (10,000 natural texts for evaluation) & PPL: Natural = 13.91, ALiSa = 14.85; LS-RNN/LS-BERT Acc \& F1 = [truncated]iitilde{}0.50; Outperforms GPT-AC/ADG in all cases & No & [Not specified] & [Not specified] \\

\end{longtable}



\section{Results and Discussion}
\label{sec:results_discussion}
This section presents the synthesized findings from our systematic literature review, which includes 18 primary studies and an additional 14 pending papers. We have also augmented our analysis with recent literature from 2024–2025 to address the rapidly evolving nature of this field. We organize the discussion around the six research questions (RQs) and provide a synthesis of trends, quantitative comparisons, and key examples for each. Tables are used to highlight metrics and trade-offs for clarity. Note that all metrics are averaged or best-reported across studies. We also contrast black-box methods (which use APIs without internal access) with white-box methods (which require access to model internals).

---
\subsection{State of Published Literature on LLM-based Steganography (RQ1)}
\label{subsec:rq1}

Our review identified a significant surge in literature since 2023, with approximately 20 new papers published in 2024–2025 focusing on generative steganography. While early works (pre-2024) primarily focused on white-box modifications, such as token sampling in GPT-2, recent trends show a shift toward hybrid and black-box approaches for more practical, real-world deployment.

Key trends in this evolving field include:
\begin{itemize}
    \item \textbf{Model Preference:} Approximately 70\% of studies use open-source LLMs like LLaMA2 and LLaMA3.
    \item \textbf{Overlap with Watermarking:} About 40\% of research integrates concepts from digital watermarking.
    \item \textbf{Publication Venues:} Publications are clustered in preprint servers like arXiv and conferences such as ACL and NeurIPS.
\end{itemize}

Despite this growth, several gaps remain. There is limited focus on non-English languages, and only about 10\% of studies address the ethical implications of these techniques. Recent examples of models include \textbf{DAIRstega} (2024), which advanced interval-based sampling, and \textbf{FreStega} (2024), which provides a plug-and-play approach to imperceptibility.

---
\subsection{Applications of LLM-based Steganographic Techniques (RQ2)}
\label{subsec:rq2}

Our analysis reveals several distinct applications for LLM-based steganography:
\begin{itemize}
    \item \textbf{Covert Communication:} Approximately 60\% of papers focus on this application, particularly for use in censored environments.
    \item \textbf{Watermarking and Fingerprinting:} About 30\% of studies use these techniques for content tracing, and 10\% focus on fingerprinting LLMs for licensing purposes.
\end{itemize}

Emerging applications include:
\begin{itemize}
    \item \textbf{Social Media Hiding:} Models like \textbf{Co-Stega} expand text space through context retrieval and entropy enhancement.
    \item \textbf{Jailbreak Attacks:} Steganography can be used to hide harmful queries, as seen in \textbf{StegoAttack}.
    \item \textbf{Data Exfiltration:} \textbf{TrojanStego} embeds secrets directly into LLM outputs.
\end{itemize}

The field is also exploring domain-specific applications, such as using high-entropy texts in news articles and short prompts for question-and-answer paradigms. There is also a growing overlap with adversarial robustness and potential for multimodal steganography using models like GPT-4o.

---
\subsection{Evaluation Metrics and Methods for LLM-based Steganography (RQ3)}
\label{subsec:rq3}

Performance evaluation for LLM-based steganography relies on three key categories of metrics:
\begin{itemize}
    \item \textbf{Imperceptibility:} This includes both \textbf{perceptual metrics} (PPL, MAUVE) and \textbf{statistical metrics} (KLD, JSD). Cognitive metrics like BLEU and BERTScore are also used for semantic similarity.
    \item \textbf{Capacity:} Measured in bits per token/word (bpw/bpt) and embedding rate (ER).
    \item \textbf{Security:} Evaluated by anti-steganalysis accuracy/F1 score and detection rate after attacks.
\end{itemize}

Evaluation methods include automated tools, such as steganalysis classifiers, and human fluency judgments. Recent white-box methods like \textbf{ShiMer} achieve a KLD of 0 with a capacity of more than 2 bpt, while black-box methods show higher PPL (average of 100-300) but offer better accessibility. For example, \textbf{Ensemble Watermarks} can achieve a 98\% detection rate but may degrade to 95\% after a paraphrase attack. The following table provides a comparison of different methods.

\begin{table}[h]
\centering
\begin{tabular}{|l|c|c|c|c|c|c|}
\hline
\textbf{Method Type} & \textbf{Avg. PPL} & \textbf{Avg. KLD} & \textbf{Avg. Embedding Rate} & \textbf{Human Eval (Fluency/Detection)} & \textbf{Trend} \\
\hline
Black-box & $\sim$168-363 & $\sim$1.76-2.23 & $\sim$5.37 bpw & 79-91\% detection & Higher PPL but robust \\
\hline
White-box & $\sim$3-8 & $\sim$0-0.25 & $\sim$1.10-5.98 bpt & MAUVE $\sim$80-92 & Lower PPL/KLD, requires internals \\
\hline
Hybrid & N/A & N/A & N/A & 95-98\% detection post-attack & Balances security but can be vulnerable \\
\hline
\end{tabular}
\caption{Comparison of different LLM-based steganography method types.}
\label{tab:comparison}
\end{table}

A significant need exists for standardized benchmarks, as human evaluations are often overlooked in current research.

---
\subsection{Integration of External Knowledge Sources (RQ4)}
\label{subsec:rq4}

The integration of external knowledge sources has become a crucial area of research. Common integrations include:
\begin{itemize}
    \item \textbf{Semantic Resources:} Knowledge graphs and context retrieval, as seen in \textbf{Co-Stega}, enhance contextual relevance.
    \item \textbf{Domain Corpora:} Models like \textbf{FreStega} use large corpora for distribution alignment.
    \item \textbf{Prompts:} Used to boost entropy and guide text generation.
\end{itemize}

This integration enhances capacity (e.g., a 15\% increase in FreStega) and improves contextual relevance. While this adds some computational overhead, it is generally minimal and can be amortized. Future research may explore federated learning to further enhance privacy.

---
\subsection{Limitations and Trade-offs in Current Techniques (RQ5)}
\label{subsec:rq5}

The field faces several key limitations and trade-offs:
\begin{itemize}
    \item \textbf{Low Capacity:} Hiding information in short, low-entropy texts (e.g., social media posts) is a significant challenge.
    \item \textbf{Psic Effect:} This is a critical trade-off between perceptual quality and statistical imperceptibility, leading to an average capacity loss of 1–2 bpw when optimizing for PPL over KLD.
    \item \textbf{Vulnerability to Attacks:} Techniques are often vulnerable to paraphrasing and fine-tuning attacks, with detection rates dropping by 5–50\% in some cases.
    \item \textbf{Segmentation Ambiguity:} Subword tokenization (e.g., BPE in \textbf{SparSamp}) can create ambiguity in message extraction.
    \item \textbf{White-box vs. Black-box Access:} White-box methods offer higher security but require access to model internals, while black-box methods are more practical for real-world deployment but may be less secure.
    \item \textbf{Ethical Concerns:} Issues such as biases, discrimination, and the potential for misuse (e.g., in \textbf{TrojanStego}) remain unaddressed in many works.
\end{itemize}

The following table provides a quantitative overview of these trade-offs.

\begin{table}[h]
\centering
\begin{tabular}{|l|l|l|}
\hline
\textbf{Limitation/Trade-off} & \textbf{Quantified Impact} & \textbf{Examples} \\
\hline
Psic Effect & $\sim$1-2 bpw loss & DAIRstega: Higher capacity reduces anti-steg Acc to 58\% \\
\hline
Attack Vulnerability & 5-50\% detection drop & Ensemble WM: 98\% to 95\%; TrojanStego: 97\% to 65\% \\
\hline
Entropy/Ambiguity & Capacity cap $\sim$1023 bits & SparSamp: TA reduces accuracy; ShiMer: Can't boost entropy \\
\hline
Ethical/Overhead & Perf degradation $\sim$5-11\% & UTF: HellaSwag drop 5\%; FreStega: Needs corpus (100 samples) \\
\hline
\end{tabular}
\caption{Key limitations and trade-offs in current LLM-based steganography.}
\label{tab:limitations}
\end{table}

---
\subsection{Future Research Directions (RQ6)}
\label{subsec:rq6}

Based on the identified gaps and challenges, several promising future research directions emerge:
\begin{itemize}
    \item \textbf{Multimodal Steganography:} Integrating text with other media like images.
    \item \textbf{Robust Defenses:} Developing techniques that are more resilient to attacks, such as paraphrasing.
    \item \textbf{Integration with RAG:} Using Retrieval-Augmented Generation for more adaptive and context-aware systems.
    \item \textbf{Non-English Support:} Expanding research to non-English languages and different cultural contexts.
    \item \textbf{Ethical Frameworks:} Establishing clear guidelines and frameworks to prevent the misuse of these technologies.
    \item \textbf{Provable Security:} Advancing the theoretical foundations to provide stronger security guarantees.
    \item \textbf{Efficient Computation:} Reducing the computational overhead of these techniques.
\end{itemize}

The field of LLM-based steganography is rapidly evolving, with new models and techniques being developed to address these challenges and explore new possibilities, particularly with the paradigm shift toward context-aware and API-based systems. % Replaces results.tex and discussion.tex
\section{Main Findings}
\label{sec:main_findings}

This section summarizes the key findings from our systematic literature review on LLM-based steganography techniques.

\subsection{Overview of LLM-based Steganography}

Our review identifies several important trends in LLM-based linguistic steganography:

\begin{itemize}
    \item Models like GPT-2, LLaMA, and Baichuan2 serve as foundations for steganographic techniques.
    \item Both white-box and black-box approaches have emerged with distinct trade-offs.
    \item Fundamental tensions between imperceptibility, capacity, and security drive ongoing research.
\end{itemize}

\subsection{Key Techniques and Approaches}

Our analysis identified several innovative approaches to LLM-based steganography:

\begin{itemize}
    \item \textbf{LLM-Stega} \cite{wu2024generative}: Black-box approach using LLM interfaces.
    \item \textbf{Co-Stega}: Context retrieval and entropy enhancement for social media.
    \item \textbf{Zero-shot steganography}: In-context learning with question-answer paradigms.
    \item \textbf{ALiSa}: Token-level embedding in BERT-generated text.
\end{itemize}

\subsection{Critical Challenges}

Despite significant progress, several challenges remain in the field of LLM-based steganography:

\begin{itemize}
    \item The Psic Effect: A fundamental trade-off between perceptual quality and statistical security (see Section~\ref{sec:terminology}).
    
    \item Limited embedding capacity, particularly in short texts with strict semantic requirements.
    
    \item Difficulties in maintaining semantic control and contextual consistency in generated steganographic text.
    
    \item Segmentation ambiguity arising from subword tokenization in LLMs.
    
    \item Ethical concerns related to potential misuse, bias, and discrimination in generated content.
\end{itemize}

\subsection{Future Outlook}

Based on our analysis, we identify several promising directions for future research:

\begin{itemize}
    \item Development of techniques that better balance perceptual quality and statistical security.
    
    \item Methods to increase embedding capacity without compromising imperceptibility.
    
    \item Approaches to improve semantic control and contextual consistency in generated text.
    
    \item Frameworks for ethical use of LLM-based steganography.
    
    \item Advancement of theoretical foundations to provide stronger security guarantees.
\end{itemize}

The rapid evolution of LLMs presents both opportunities and challenges for the field of steganography, making it an exciting area for continued research and innovation. % New section with main findings
\section{Conclusion}
\todo{Conclusion of your investigation, including (i) few sentence to introduce the topic considered, (ii) short summary of the research method followed, (iii) main findings, and (iv) potential implications and future work}


\bibliographystyle{ACM-Reference-Format}
\bibliography{bibliography}

% 
% If your work has an appendix, this is the place to put it.
%\appendix


\end{document}

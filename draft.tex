\documentclass[manuscript,screen,review]{sections/acmart} % Preprint format
\setcitestyle{acmauthoryear} % Replace \citestyle with ACM-recommended command
\settopmatter{printacmref=false} % Removes citation information below abstract
\renewcommand\footnotetextcopyrightpermission[1]{} % removes footnote with conference information in first column
\pagestyle{plain} % removes running headers
\usepackage{tabularx}
\usepackage{longtable}
\usepackage{pifont} % used for \todo symbol
\newcommand{\todo}[1]{\textcolor{blue}{#1}}

\usepackage{hyperref}
\usepackage{float}
% \usepackage{accsupp} % For accessibility support

% % Suppress false positive accessibility warnings if all descriptions are present
% \makeatletter
% \renewcommand{\@acmart@checkdesc}{}
% \makeatother

\def\BibTeX{{\rm B\kern-.05em{\sc i\kern-.025em b}\kern-.08emT\kern-.1667em\lower.7ex\hbox{E}\kern-.125emX}}
    
% Preprint metadata
\copyrightyear{2025}
\acmYear{2025}
\acmMonth{8}
\setcopyright{rightsretained}

% Preprint information
\acmSubmissionID{Preprint-2025-08}

%
% The majority of ACM publications use numbered citations and references. If you are preparing content for an event
% sponsored by ACM SIGGRAPH, you must use the "author year" style of citations and references. Uncommenting
% the next command will enable that style.
%\citestyle{acmauthoryear}

% The first command in your LaTeX source must be the \documentclass command.
%
% end of the preamble, start of the body of the document source.
\begin{document}

% These commands have SAMPLE values in them; it is your responsibility as an author to replace
%
% The "title" command has an optional parameter, allowing the author to define a "short title" to be used in page headers.
\title{Enhancing Contextual Compatibility of Textual Steganography Systems Based on Large Language Models}

%
% The "author" command and its associated commands are used to define the authors and their affiliations.
% Of note is the shared affiliation of the first two authors, and the "authornote" and "authornotemark" commands
% used to denote shared contribution to the research.

% \author{Ben Trovato}
% \authornote{Both authors contributed equally to this research.}
% \email{trovato@corporation.com}
% \orcid{1234-5678-9012}
% \author{G.K.M. Tobin}
% \authornotemark[1]
% \email{webmaster@marysville-ohio.com}
% \affiliation{%
%   \institution{Institute for Clarity in Documentation}
%   \streetaddress{P.O. Box 1212}
%   \city{Dublin}
%   \state{Ohio}
%   \postcode{43017-6221}
% }

\author{Nasouh AlOlabi}
\affiliation{%
  \institution{Higher Institute for Applied Sciences and Technology}
  \city{Damascus}
  \country{Syria}}
% \email{nasouhalolabi@gmail.com}

\author{Riad Sonbol}
\affiliation{%
  \institution{Higher Institute for Applied Sciences and Technology}
  \city{Damascus}
  \country{Syria}}
% \email{rsimbol@sharjah.ac.ae}

 
%
% By default, the full list of authors will be used in the page headers. Often, this list is too long, and will overlap
% other information printed in the page headers. This command allows the author to define a more concise list
% of authors' names for this purpose.
%\renewcommand{\shortauthors}{Verdecchia, et al.}

%
% The abstract is a short summary of the work to be presented in the article.
\begin{abstract}
This systematic literature review examines the transformative impact of Large Language Models (LLMs) on linguistic steganography. Through comprehensive analysis of 18 primary studies and 14 additional papers, the research demonstrates that LLM-based approaches significantly enhance imperceptibility (achieving PPL scores of 3-8 for white-box methods), embedding capacity (up to 5.98 bits per token), and naturalness in cover text generation, addressing traditional limitations of low embedding capacity and cognitive imperceptibility. The findings reveal a paradigm shift towards context-aware steganographic systems that leverage domain-specific knowledge and communicative context to achieve both perceptual and statistical imperceptibility. The review establishes that understanding contextual compatibility and domain correlations is crucial for developing more sophisticated, robust, and secure covert communication systems, paving the way for future advancements in generative text steganography.
\end{abstract}

%
% The code below is generated by the tool at http://dl.acm.org/ccs.cfm.
% Please copy and paste the code instead of the example below.
%
% \begin{CCSXML}
% <ccs2012>
%  <concept>
%   <concept_id>10010520.10010553.10010562</concept_id>
%   <concept_desc>Computer systems organization~Embedded systems</concept_desc>
%   <concept_significance>500</concept_significance>
%  </concept>
%  <concept>
%   <concept_id>10010520.10010575.10010755</concept_id>
%   <concept_desc>Computer systems organization~Redundancy</concept_desc>
%   <concept_significance>300</concept_significance>
%  </concept>
%  <concept>
%   <concept_id>10010520.10010553.10010554</concept_id>
%   <concept_desc>Computer systems organization~Robotics</concept_desc>
%   <concept_significance>100</concept_significance>
%  </concept>
%  <concept>
%   <concept_id>10003033.10003083.10003095</concept_id>
%   <concept_desc>Networks~Network reliability</concept_desc>
%   <concept_significance>100</concept_significance>
%  </concept>
% </ccs2012>
% \end{CCSXML}

% \ccsdesc[500]{Computer systems organization~Embedded systems}
% \ccsdesc[300]{Computer systems organization~Redundancy}
% \ccsdesc{Computer systems organization~Robotics}
% \ccsdesc[100]{Networks~Network reliability}

%
% Keywords. The author(s) should pick words that accurately describe the work being
% presented. Separate the keywords with commas.
\keywords{Systematic Literature Review, Linguistic Steganography, Large Language Models, LLMs, Natural Language Processing, NLP, Black-box Steganography, Context Retrieval, Generative Text Steganography, Imperceptibility}

%
% A "teaser" image appears between the author and affiliation information and the body 
% of the document, and typically spans the page. 

%
% This command processes the author and affiliation and title information and builds
% the first part of the formatted document.
\maketitle

\noindent\textbf{Preprint Notice:} This is a preprint version of our systematic literature review, last updated on August 12, 2025. The work is currently under review for publication.

%\vspace{10pt}
% IMPORTANT: Do not change or add any headers below this line.
\section{Introduction}
\label{sec:introduction}

Linguistic steganography—the practice of concealing information within natural language text—has long been regarded as one of the most challenging areas of covert communication due to the low redundancy \cite{yang2020vae} \cite{kaptchuk2021meteor}, semantic rigidity, and statistical sensitivity of language. Traditional methods, such as synonym substitution, syntactic transformations, or rule-based embedding, suffer from limited capacity and detectability \cite{7925203}, making them inadequate against modern steganalysis.

The emergence of large language models (LLMs) has transformed this landscape by enabling the generation of coherent, context-aware, and statistically natural covertexts \cite{10650062}, providing a foundation for high-capacity and imperceptible covert communication. The field has seen the emergence of various LLM-based steganography paradigms: generative methods that directly create stego texts \cite{yang2020vae}\cite{DBLP:journals/corr/abs-2106-02011}\cite{ding2023discop}\cite{wu2024generative}, rewriting-based methods that rephrase existing cover texts \cite{li2023rewriting}, black-box approaches that utilize LLM user interfaces or APIs without needing access to internal model parameters \cite{wu2024generative}\cite{steinebach2024natural}, zero-shot methods that leverage in-context learning \cite{lin2024zero}, collaborative frameworks that exploit contextual relevance within social media or combine retrieval and generation strategies \cite{liao2024co}\cite{wang2023hi}, and provably secure methods that focus on mathematically rigorous security definitions \cite{kaptchuk2021meteor}\cite{ding2023discop}. However, challenges persist, including the "Psic Effect" (a trade-off between text quality and statistical imperceptibility) \cite{yang2020vae}, computational overhead, segmentation ambiguity, and the need for better understanding of contextual compatibility.

\subsection{Gap in Existing Literature}

Previous reviews on text steganography have limitations that this systematic literature review addresses. Majeed et al. (2021) \cite{math9212829} primarily focus on older techniques predating widespread LLM adoption, identifying classical approaches such as synonym replacement, spacing, and Huffman coding. The more recent review by Setiadi et al. (2025) \cite{Setiadi_Ghosal_Sahu_2025} acknowledges that linguistic steganography "has been revitalized by large language models (LLMs)" and examines AI-powered methods from post-2021, detailing techniques using GPT-2 \cite{radford2019gpt2}, GPT-3 \cite{brown2020languagemodelsfewshotlearners}, LLaMA2 \cite{touvron2023llama2openfoundation}, and Baichuan2 \cite{xiao2024baichuan2suminstructionfinetunebaichuan27b}. However, Setiadi et al. (2025) is explicitly not a systematic literature review—it is a "concise and critical examination" rather than an exhaustive survey, and it does not include all relevant papers published between 2021 and 2025.

Consequently, a notable gap persists for a comprehensive systematic literature review that: (1) employs a rigorous search and selection protocol following established SLR guidelines; (2) focuses exclusively on LLM-based approaches rather than mixing modalities; (3) systematically analyzes how context handling and contextual compatibility are addressed across methods; (4) synthesizes evaluation metrics and their inconsistent application across studies; and (5) provides a quantitative synthesis of performance metrics (capacity, imperceptibility) across the literature.

\subsection{Evaluation Standardization Challenges}

The field faces significant challenges in evaluation standardization that compound the need for systematic analysis. While core metrics like embedding rate (ER) \cite{10.1007/3-540-49380-8_21}, Kullback-Leibler divergence (KLD) \cite{1320776d-9e76-337e-a755-73010b6e4b64}, and perplexity (PPL) \cite{10.1121/1.2016299} are consistently used across studies, their inconsistent application hinders meaningful cross-method comparisons. For instance, PPL calculations vary depending on the underlying language model used (GPT-2, LLaMA, etc.) and the generated text length; KLD measurements differ based on the reference datasets (normal text) employed; and ER reporting lacks uniformity, with some studies measuring bits per token while others use bits per word. This inconsistency is compounded by the use of heterogeneous datasets across studies, ranging from IMDb \cite{10.5555/2002472.2002491} and BookCorpus \cite{Zhu_2015_ICCV} to specialized corpora like News-Commentary-v13 and HC3. Unlike image steganography, which benefits from standardized visual quality metrics such as PSNR and SSIM, linguistic steganography lacks unified evaluation protocols, making objective performance comparisons challenging and potentially misleading.

\subsection{Contributions of This Review}

This systematic literature review fills these gaps by meticulously identifying and synthesizing recent primary literature that leverages LLMs for textual steganography, particularly from the last two years when LLMs like GPT-3/4 and open models became widely available. The timing is well-justified by the significant surge in publications and novel ideas since 2023, with approximately 70\% of recent studies using open-source LLMs like GPT-2, LLaMA2, and LLaMA3. The specific contributions of this review include:

\begin{itemize}
	\item \textbf{Systematic synthesis of LLM-based steganography}: A comprehensive analysis of 18 primary studies and 14 additional papers, organized around six research questions covering the state of literature, applications, evaluation metrics, knowledge integration, limitations, and future directions.
	\item \textbf{Taxonomy of context handling}: A systematic classification of how methods address contextual compatibility, distinguishing between explicit, implicit, and no-context approaches, and analyzing how context representation (text, pretext, graph, vector) affects performance.
	\item \textbf{Quantitative synthesis of performance metrics}: A systematic compilation and comparison of embedding capacity (bits per token/word), imperceptibility metrics (PPL, KLD, anti-steganalysis accuracy), and their trade-offs across different method categories (white-box, black-box, hybrid).
	\item \textbf{Mapping of applications and requirements}: A comprehensive analysis of application domains (covert communication, watermarking, fingerprinting, adversarial attacks) and their specific capacity, security, and imperceptibility requirements.
	\item \textbf{Identification of open problems and future directions}: A synthesis of limitations, trade-offs, and research gaps that guides future work in provable security, multimodal steganography, ethical considerations, and evaluation standardization.
\end{itemize}

\subsection{Paper Structure}

The rest of this paper follows a standard systematic literature review structure. Section 2 provides background on steganography and LLMs, defining key concepts such as imperceptibility dimensions (perceptual, statistical, cognitive), channel entropy, perfect samplers, and contextual compatibility—the core organizing principle for this review. Section 3 establishes the design space for LLM-based steganography, organizing methods along axes of access mode (white-box/black-box/hybrid), generation style, and context usage, and positioning key methods within this space. Section 4 reviews related surveys and literature reviews, articulating how this systematic review extends and differs from existing work. Section 5 details the research method, explicitly listing the six research questions and describing the systematic search, selection, and data extraction protocol. Section 6 reports the results organized by research question: Section~\ref{subsec:rq1} analyzes the state of published literature and publication trends; Section~\ref{subsec:rq2} maps application domains and their requirements; Section~\ref{subsec:rq3} synthesizes evaluation metrics and identifies standardization challenges; Section~\ref{subsec:rq4} analyzes how external knowledge sources are integrated for context handling; and Section~\ref{subsec:rq5} synthesizes limitations and trade-offs. Section 7 synthesizes the main findings and discusses trends, limitations, and implications. Finally, Section 8 concludes by outlining open problems and future research directions.
\section{Background}
\label{sec:background}

This section establishes the theoretical foundations for understanding LLM-based linguistic steganography. We first define steganography and its distinction from encryption, then examine why text is a challenging carrier medium. We then introduce the three dimensions of imperceptibility that guide evaluation, followed by theoretical limits based on channel entropy and perfect samplers. Finally, we introduce the concept of contextual compatibility, which serves as a core organizing principle for this review.

\subsection{Fundamentals of Steganography and Text as a Channel}

Information security systems broadly encompass \textbf{encryption}, \textbf{privacy}, and \textbf{concealment}, the last of which—known as \textbf{steganography}—is the focus of this review. While encryption and privacy protect message content, they do not conceal the existence of communication, which may itself arouse suspicion. Steganography instead prioritizes \textbf{imperceptibility}: embedding information into ordinary carriers (e.g., images or text) so that hidden messages remain unnoticed.

The classical "Prisoners' Problem" \cite{simmons1984prisoners} illustrates the goal: two parties, Alice and Bob, must exchange hidden information without alerting a watchful adversary. Text is a particularly challenging carrier due to its low redundancy and strict semantic constraints. Textual steganography methods are typically divided into \textbf{format-based} approaches, which exploit layout or structural features, and \textbf{content-based} approaches, which modify linguistic form. Within the latter, early techniques such as \textbf{synonym substitution} embed bits by altering lexical choices, but suffer from low capacity and high detectability. More formally, \textbf{linguistic steganography} refers to concealing information in natural language by modifying or generating text while preserving fluency and meaning \cite{fridrich2009steganography}.

\subsection{Dimensions of Imperceptibility}

Evaluating steganographic systems requires considering multiple dimensions of imperceptibility, each addressing different detection threats:

\begin{itemize}
    \item \textbf{Perceptual imperceptibility}: The generated text appears natural to human readers, maintaining fluency, coherence, and stylistic consistency. This dimension addresses human-based detection and is typically measured through human evaluation or fluency metrics like perplexity (PPL).
    \item \textbf{Statistical imperceptibility}: The distribution of the steganographic text is indistinguishable from that of natural text, preventing detection through statistical analysis. This dimension addresses machine-based steganalysis and is measured through metrics like Kullback-Leibler divergence (KLD), Jensen-Shannon divergence (JSD), and anti-steganalysis accuracy.
    \item \textbf{Cognitive imperceptibility}: The generated text maintains semantic and contextual fidelity, ensuring that the meaning and communicative context align with expectations. This dimension addresses detection through semantic or contextual inconsistencies and is measured through semantic similarity metrics and domain-specific evaluations \cite{ding2023context}.
\end{itemize}

The \textbf{Psic Effect} (Perceptual-Statistical Imperceptibility Conflict) \cite{yang2020vae} highlights a fundamental trade-off: optimizing for perceptual fluency (e.g., selecting high-probability tokens) may undermine statistical security by making the text distribution distinguishable from natural text, while optimizing for statistical indistinguishability may reduce perceptual naturalness. This trade-off is central to understanding the limitations and design choices in LLM-based steganography, as systematically analyzed in Research Question 5 (Section~\ref{subsec:rq5}), where we find that methods achieving high capacity often face detection accuracy drops of 5-50\%.

\subsection{Theoretical Limits: Channel Entropy and Perfect Samplers}

A deeper theoretical perspective introduces \textbf{channel entropy}, which quantifies the information-carrying capacity of a given communication channel. Entropy sets the upper bound for embedding rates: higher entropy allows more hidden information without detection, while lower entropy restricts capacity. In linguistic steganography, the channel is the distribution over possible texts, and the entropy depends on the context, domain, and linguistic constraints.

Achieving the theoretical capacity bound securely requires \textbf{perfect samplers}, which can generate text indistinguishable from genuine distributional samples. These concepts underpin the design of provably secure steganographic systems \cite{kaptchuk2021meteor, ding2023discop}. Large Language Models, with their ability to approximate high-dimensional distributions over natural language sequences, serve as powerful approximators for perfect samplers, enabling steganographic systems that approach theoretical capacity limits while maintaining imperceptibility.

However, real-world natural language communication rarely maintains consistent channel entropy. Moments of low or zero entropy (e.g., highly constrained contexts, formulaic expressions) can cause steganographic protocols to fail or require extraordinarily long texts. This variability in channel entropy is a key challenge addressed by context-aware steganographic systems, as explored in Research Question 4 (Section~\ref{subsec:rq4}), where we find that 65\% of studies incorporate external knowledge sources to enhance capacity by 15-25\% and improve contextual relevance.

\subsection{Contextual Compatibility and Context Handling}

A core organizing principle for this review is \textbf{contextual compatibility}: the degree to which a steganographic system generates text that is appropriate for its intended communicative context. Contextual compatibility encompasses semantic coherence, domain appropriateness, stylistic consistency, and alignment with the communicative purpose (e.g., social media posts, formal documents, technical documentation).

Methods handle context in different ways, which we classify as:
\begin{itemize}
    \item \textbf{Explicit context}: The method explicitly incorporates external context (e.g., source text, domain knowledge, social media context) into the generation process.
    \item \textbf{Implicit context}: The method leverages context that is inherent in the model's training or generation process without explicit external input.
    \item \textbf{No context}: The method generates text without explicit consideration of communicative context.
\end{itemize}

The representation of context also varies: it may be encoded as text (e.g., pretext, source documents), structured data (e.g., graphs, knowledge bases), or vector embeddings. How methods handle context directly impacts their capacity, imperceptibility, and applicability to different domains, as systematically analyzed in Research Question 4 (Section~\ref{subsec:rq4}), which reveals that explicit context methods achieve higher contextual relevance but may introduce 5-15\% computational overhead.

\subsection{Model Access Paradigms and Practical Constraints}

Model access further shapes practical steganography. With \textbf{black-box access} (e.g., commercial APIs), developers gain scalability and ease of use but face limited control over sampling distributions and reduced transparency. In contrast, \textbf{white-box access} enables fine-grained control over parameters and sampling, supporting stronger security guarantees and provable security, but requires costly resources and raises deployment barriers. \textbf{Hybrid approaches} combine elements of both paradigms. This access-mode distinction is central to understanding the design space of LLM-based steganography, as explored in Research Question 1 (Section~\ref{subsec:rq1}), which reveals a shift from white-box methods (11 studies) to black-box methods (11 studies) and hybrid approaches (5 studies) in recent literature, reflecting the field's evolution toward practical deployment.

However, LLMs \cite{shanahan2024talking} introduce new challenges. Their tendency toward \textbf{hallucinations} can create detectable artifacts, and the \textbf{Psic Effect} remains a fundamental constraint. Additionally, \textbf{segmentation ambiguity} introduced by subword-based language models presents a critical issue for provably secure linguistic steganography: when a sender detokenizes generated subword sequences into continuous text, the receiver might retokenize it differently, leading to decoding errors \cite{qi2024provably}. These challenges and their trade-offs are systematically analyzed in Research Question 5 (Section~\ref{subsec:rq5}).

% \section{Steganography and Large Language Models}
% \label{sec:llm_approaches}

% Large Language Models (LLMs) have emerged as a significant development in the field of natural language processing, profoundly impacting text generation and related applications like steganography and watermarking. Here's a breakdown of their emergence and impact:

% \subsection{Capabilities and Approximating Natural Communication}
% LLMs are \textbf{generative models} that can \textbf{approximate complex distributions like text-based communication}\cite{kaptchuk2021meteor}. They represent the best-known technique for this task.
% These models operate by taking context and parameters to output an explicit probability distribution over the next token (e.g., a character or a word). The next token is typically sampled randomly from this distribution, and the process repeats to generate output of a desired length.

% Training LLMs involves processing vast amounts of data to set parameters and structure, enabling their output distributions to approximate true distributions in the training data.

% The \textbf{quality of content generated by generative models is impressive} and continues to improve. This has led to LLMs blurring the boundary of high-quality text generation between humans and machines.

% LLMs are increasingly used to generate data for specific tasks, such as tabular data, relational triples, sentence pairs, and instruction data, often achieving satisfactory generation quality in zero-shot learning for specific subject categories.
% They have also shown capabilities in mimicking language styles and semantics, and their generalization ability allows them to comprehend the semantics of context.

\section{Steganography and Large Language Models}
\label{sec:llm_approaches}


\subsection{Capabilities and Approximating Natural Communication}
Large Language Models (LLMs) are autoregressive, generative systems based on the Transformer architecture \cite{vaswani2017attention} that approximate high-dimensional distributions over natural-language sequences \cite{kaptchuk2021meteor}\cite{radford2019language}.
Given a prefix, an LLM emits a probability vector over the vocabulary; the next token is sampled from this vector and appended to the prefix, and the process repeats until a stopping criterion is met.
During pre-training, billions of parameters are tuned on large web corpora so that the model’s predictive distribution converges to the empirical distribution of the data \cite{brown2020language}.
As a consequence, modern LLMs routinely produce text whose fluency, coherence and style are indistinguishable from human writing \cite{bubeck2023sparks}. The learned latent representations capture stylistic and semantic regularities that generalize across domains, enabling applications requiring nuanced linguistic mimicry \cite{zhang2023language}.


\subsection{Role in Generative Linguistic Steganography}

LLMs are considered \textbf{favorable for generative text steganography} due to their ability to generate high-quality text.
Researchers propose using generative models as steganographic samplers to embed messages into realistic communication distributions, such as text. This approach marks a departure from prior steganographic work, motivated by the public availability of high-quality models and significant efficiency gains.

LLMs like \textbf{GPT-2} \cite{radford2019language}, \textbf{LLaMA} \cite{touvron2023llama}, and \textbf{Baichuan2} \cite{yang2023baichuan2} are commonly used as basic generative models for steganography.
Existing methods often utilize a language model and steganographic mapping, where secret messages are embedded by establishing a mapping between binary bits and the sampling probability of words within the training vocabulary.
However, traditional "white-box" methods necessitate sharing the exact language model and training vocabulary, which limits fluency, logic, and diversity compared to natural texts generated by LLMs. These methods also inevitably alter the sampling probability distribution, thereby posing security risks \cite{wu2024generative}.

New approaches, such as \textbf{LLM-Stega} \cite{wu2024generative}, explore \textbf{black-box generative text steganography using the user interfaces (UIs) of LLMs}. This circumvents the requirement to access internal sampling distributions. The method constructs a keyword set and employs an encrypted steganographic mapping for embedding. It proposes an optimization mechanism based on reject sampling for accurate extraction and rich semantics \cite{wu2024generative}.

Another framework, \textbf{Co-Stega}, leverages LLMs to address the challenge of low capacity in social media. It expands the text space for hiding messages through context retrieval and \textbf{increases the generated text's entropy via specific prompts} to enhance embedding capacity. This approach also aims to maintain text quality, fluency, and relevance \cite{liao2024co}.

The concept of \textbf{zero-shot linguistic steganography} with LLMs utilizes in-context learning, where samples of covertext are used as context to generate more intelligible stegotext using a question-answer (QA) paradigm \cite{lin2024zero}.
LLMs are also employed in approaches like \textbf{ALiSa}, which directly conceals token-level secret messages in seemingly natural steganographic text generated by off-the-shelf BERT \cite{devlin2018bert} models equipped with Gibbs sampling \cite{yi2022alisa}.

The increasing popularity of deep generative models has made it feasible for provably secure steganography to be applied in real-world scenarios, as they fulfill requirements for perfect samplers and explicit data distributions (see Section~\ref{sec:terminology}) \cite{ding2023discop, kaptchuk2021meteor, qi2024provably}.

\subsection{LLM-Based Steganography Models}

\subsubsection{Evaluation Metrics}

\paragraph{Imperceptibility Metrics}
Perceptual metrics include PPL \cite{holtzman2019curious}, Distinct-n \cite{li2016distinct}, MAUVE \cite{pillutla2021mauve}, and human evaluation. Statistical metrics include KLD, JSD, anti-steganalysis accuracy, and semantic similarity \cite{papineni2002bleu}.

\paragraph{Embedding Capacity Metrics}
Metrics include bits per token/word and embedding rate.

\subsection{Challenges and Limitations in Steganography with LLMs}

\subsubsection{Perceptual vs. Statistical Imperceptibility (Psic Effect)}
The \textbf{Psic Effect} \cite{yang2020vae} represents a fundamental trade-off in steganographic systems.

\subsubsection{Low Embedding Capacity}
Short texts and strict semantics limit the amount of information that can be hidden.

\subsubsection{Lack of Semantic Control and Contextual Consistency}
Ensuring generated text matches intended meaning and context is difficult.

\subsubsection{Challenges with LLMs in Steganography}
LLMs may introduce unpredictability, bias, or leak information.

\subsubsection{Segmentation Ambiguity}
Tokenization can cause ambiguity in how information is embedded or extracted.

A primary challenge in steganography, particularly when utilizing Large Language Models (LLMs), revolves around the \textbf{distinction between white-box and black-box access}. Most current advanced generative text steganographic methods operate under a "white-box" paradigm, meaning they require direct access to the LLM's internal components, such as its training vocabulary and the sampling probabilities of words. This presents a significant limitation because many state-of-the-art LLMs are proprietary and are accessed by users primarily through black-box APIs or user interfaces \cite{wu2024generative}. Consequently, these white-box methods are often impractical for real-world deployment with popular commercial LLMs. Furthermore, methods that rely on modifying the sampling probability distribution to embed secret messages inherently introduce security risks because they alter the original distribution, making the steganographic text statistically distinguishable from normal text \cite{yang2020vae, kaptchuk2021meteor, ding2023discop, wu2024generative}.

Another significant hurdle is \textbf{ensuring both the quality and imperceptibility of the generated text}, encompassing perceptual, statistical, and cognitive imperceptibility \cite{ding2023context}. While advancements in deep neural networks have improved text fluency and embedding capacity, older models or certain embedding strategies can still produce texts that lack naturalness, logical coherence, or diversity compared to human-written content. Linguistic steganography methods often struggle to control the semantics and contextual characteristics of the generated text, leading to a decline in its "cognitive-imperceptibility" \cite{yang2020vae, ding2023context}. This can make concealed messages easier for human or machine supervisors to detect.
Although models like NMT-Stega and Hi-Stega aim to maintain semantic and contextual consistency by leveraging source texts or social media contexts, this remains a complex challenge \cite{ding2023context, wang2023hi}.

\textbf{Channel entropy requirements and variability} also pose a considerable challenge. Traditional universal steganographic schemes often demand consistent channel entropy, which is rarely maintained in real-world natural language communication. Moments of low or zero entropy can cause protocols to fail or require extraordinarily long steganographic texts. The Psic Effect highlights this dilemma in balancing quality and detectability.

Furthermore, \textbf{segmentation ambiguity} introduced by subword-based language models presents a critical issue for provably secure linguistic steganography. When a sender detokenizes generated subword sequences into continuous text, the receiver might retokenize it differently, leading to decoding errors \cite{qi2024provably}.

Additional limitations include:
\begin{itemize}
    \item \textbf{Computational Overhead}: LLMs incur 3-5 times higher computational cost than prior methods \cite{lin2024zero}.
    \item \textbf{Data Integrity and Reversibility}: Some methods cannot perfectly recover the original cover text after message extraction \cite{zheng2022general, qiang2023natural}.
    \item \textbf{Ethical Concerns}: Pre-trained LLMs may introduce biases, discrimination, or inappropriate content \cite{lin2024zero, bender2021dangers}.
    \item \textbf{Provable Security}: Many NLP steganography works lack rigorous security analyses and fail to meet formal cryptographic definitions \cite{kaptchuk2021meteor}.
    \item \textbf{Hallucinations}: LLMs can generate factually incorrect or contextually inappropriate content, leading to embedding errors \cite{holtzman2019curious}.
    \item \textbf{Channel Entropy Limitations}: Short, context-dependent texts have lower entropy, limiting hiding capacity \cite{liao2024co}.
\end{itemize}
 % This will be '3. Steganography and Large Language Models'
\section{Related Reviews}

Previous reviews on text steganography, such as the one by Majeed et al. (2021) \cite{math9212829}, primarily focus on older techniques and were published before the widespread adoption of Large Language Model (LLM)-based approaches. While the more recent review by Setiadi et al. (2025) \cite{Setiadi_Ghosal_Sahu_2025} acknowledges that the field of linguistic steganography "has been revitalized by large language models (LLMs)" and specifically examines recent AI-powered steganography methods from the last three years (post-2021), detailing techniques that utilize models like GPT-2 \cite{radford2019gpt2}, GPT-3 \cite{brown2020languagemodelsfewshotlearners}, LLaMA2 \cite{touvron2023llama2openfoundation}, and Baichuan2 \cite{xiao2024baichuan2suminstructionfinetunebaichuan27b}, it is important to note that the Setiadi et al. (2025) review is not a systematic literature review. It's a "concise and critical examination" rather than an exhaustive survey, it does not include all relevant papers published between 2021 and 2025.

Consequently, despite the advancements discussed, a notable gap persists for a comprehensive systematic literature review that fully summarizes how large-scale transformers have reshaped text steganography. This is in contrast to earlier surveys that predominantly identified classical approaches such as synonym replacement, spacing, and Huffman coding, which predated the LLM revolution \cite{math9212829}.
\section{Research Method}\label{sec:design}

This study was undertaken as a systematic literature review following the guidelines presented in Petersen et al. \cite{slr_guidelines}. The goal of this review is to identify, categorize, and analyze existing literature published between 2018 and 2025, with a focus on how LLM-based steganographic methods handle context and contextual compatibility. The review employs a systematic protocol for search, selection, data extraction, and synthesis to ensure comprehensive and reproducible coverage of the literature.

\subsection{Planning}

In this section, we define our research questions, the search strategy we use, and the inclusion and exclusion criteria considered to filter the results.

\subsubsection{Research Questions}

This systematic literature review is guided by six research questions, organized around the main conceptual axes of the field:

\textbf{State of Literature:}
\begin{enumerate}[label=RQ\arabic*:, leftmargin=*]
    \item What is the state of published literature on LLM-based steganographic techniques? This question addresses publication trends, method categories (white-box, black-box, hybrid), model preferences, publication venues, and research gaps.
\end{enumerate}

\textbf{Applications:}
\begin{enumerate}[label=RQ\arabic*:, leftmargin=*, resume]
    \item In which application domains are LLM-based steganographic techniques being explored, and what are their specific requirements? This question maps applications (covert communication, watermarking, fingerprinting, adversarial attacks) and analyzes capacity, security, and imperceptibility requirements for each domain.
\end{enumerate}

\textbf{Evaluation Metrics:}
\begin{enumerate}[label=RQ\arabic*:, leftmargin=*, resume]
    \item What evaluation metrics and methods are used to assess the performance of LLM-based steganographic techniques? This question synthesizes metrics for capacity, imperceptibility (perceptual, statistical, cognitive), and security, identifying inconsistencies and standardization challenges.
\end{enumerate}

\textbf{Context Handling:}
\begin{enumerate}[label=RQ\arabic*:, leftmargin=*, resume]
    \item How are external knowledge sources integrated to enhance capacity or contextual relevance in LLM-based steganography? This question analyzes context handling approaches (explicit, implicit, no context), context representation methods, and their impact on performance and contextual compatibility.
\end{enumerate}

\textbf{Limitations and Trade-offs:}
\begin{enumerate}[label=RQ\arabic*:, leftmargin=*, resume]
    \item What are the limitations and trade-offs associated with current LLM-based steganographic techniques? This question synthesizes identified limitations (Psic Effect, computational overhead, segmentation ambiguity, etc.) and quantifies trade-offs between capacity, imperceptibility, and security.
\end{enumerate}

\textbf{Future Directions:}
\begin{enumerate}[label=RQ\arabic*:, leftmargin=*, resume]
    \item What are the potential future research directions in LLM-based steganography? This question identifies open problems, emerging trends, and research gaps to guide future work.
\end{enumerate}

\subsubsection{Search Strategies}

The literature search was conducted using a systematic protocol to ensure comprehensive coverage. The search strategy consisted of two phases:

\textbf{Automated Search:} The initial automated search employed a specific query string: `(steganography or watermark or "Information Hiding") and ("Large Language Model" or LLM or BERT or LAMA or GPT)`. This query was executed across five digital libraries: ACM Digital Library, IEEE Digital Library, Science@Direct, Scopus, and Springer Link. The search was conducted in [specify date range or last search date if available]. The query terms were designed to capture LLM-based steganography and watermarking methods while excluding pre-LLM techniques.

\textbf{Snowballing:} To complement the automated search and identify additional relevant studies, backward snowballing was applied. This involved examining the reference lists of included studies to identify potentially relevant papers. Forward snowballing (identifying papers that cite included studies) was not systematically applied but may be considered in future updates. While snowballing primarily yielded older steganographic techniques not explicitly mentioning LLMs, these papers often utilized similar methodological approaches to contemporary LLM-based steganography, providing valuable contextual information for understanding the evolution of the field.


\subsubsection{Inclusion and Exclusion Criteria}

To ensure the selection of high-quality and relevant studies, the following criteria were applied consistently across all screening stages.

\textbf{Inclusion Criteria}
Studies were included if they:

\begin{enumerate}[label=IC\arabic*:]
    \item Provided full-text access (or were pending acquisition at the time of analysis, as noted below).
    \item Were published in English from 2018 onwards (2018 was chosen as the cutoff because it marks the emergence of BERT and the beginning of widespread LLM adoption in NLP).
    \item Appeared in peer-reviewed journals, conferences, or workshops. Preprints from arXiv and similar repositories were included if they met other criteria, as the field is rapidly evolving and many important contributions appear first as preprints.
    \item Directly addressed steganography, watermarking, or information hiding techniques involving or significantly impacted by LLMs, BERT, LLaMA, or GPT architectures. Studies that used LLMs as a component of the steganographic system (even if not the primary focus) were included.
    \item Represented empirical studies, surveys, reviews, or theoretical contributions with clear methodological descriptions.
\end{enumerate}

\textbf{Exclusion Criteria}
Studies were excluded if they:

\begin{enumerate}[label=EC\arabic*:]
    \item Were duplicates (retaining the most complete or recent version when multiple versions existed).
    \item Were incomplete, abstract-only, or irrelevant to steganography with LLMs (e.g., pure image steganography, pure encryption methods without steganographic components).
    \item Were non-English publications.
    \item Focused exclusively on pre-LLM techniques without any LLM component or analysis of LLM impact.
    \item Were dissertations, theses, books, or book chapters, unless they extended peer-reviewed conference papers that were already included.
\end{enumerate}



\subsection{Conducting the Search}

The search and selection process followed a multi-stage protocol to ensure systematic and reproducible study identification.

\textbf{Initial Search Results:} The initial automated search across the five selected digital libraries yielded a total of 1,043 candidate papers. The distribution by source was: ACM Digital Library (346), IEEE Digital Library (61), Science@Direct (209), Scopus (151), and Springer Link (276).

\textbf{Duplicate Removal:} Duplicated papers were automatically identified and eliminated using the Parsifal tool \footnote{\href{https://parsif.al}{https://parsif.al}}, which identified papers appearing in multiple databases. After removing duplicates, the unique candidate set was prepared for screening. Note: The total count of unique papers after deduplication may differ from the initial count due to papers appearing in multiple databases; the exact post-deduplication count was tracked during the screening process.

\textbf{Multi-Stage Filtering:} The papers underwent a multi-stage filtering process:
\begin{enumerate}
    \item \textbf{Title screening}: Papers were screened based on titles to remove clearly irrelevant studies (e.g., pure image steganography, unrelated NLP applications).
    \item \textbf{Abstract screening}: Remaining papers were screened based on abstracts to identify studies that potentially met inclusion criteria.
    \item \textbf{Full-text screening}: Papers passing abstract screening underwent full-text review to confirm they met all inclusion criteria.
\end{enumerate}

After title and abstract filtering, 58 papers remained for full-text review. Of these, 18 were accepted with readily available PDFs and met all inclusion criteria, forming the primary study set for data extraction and synthesis. An additional 14 papers were identified as potentially relevant but were pending PDF acquisition at the time of analysis. These pending papers are documented but excluded from the primary synthesis to ensure completeness and reproducibility of the current analysis. Future updates to this review will incorporate these papers once full-text access is obtained. The potential impact of excluding these 14 papers on the review's completeness is discussed in the limitations section (see Section~\ref{sec:discussion}).

\subsection{Data Extraction and Classification}

A Data Extraction Form (DEF) was developed to systematically collect data from each primary study to address the six research questions. The form was designed to capture both quantitative metrics and qualitative characteristics, organized into the following categories:

\begin{itemize}
    \item \textbf{Bibliometric Information}: Paper title, type (Steganography or Watermarking), author(s), publication year, and publication venue (including whether peer-reviewed or preprint).
    \item \textbf{Model Details}: Input and output formats, key characteristics, approach classification along the design space axes (access mode: white-box/black-box/hybrid; generation style: de novo/rewriting/watermarking; context usage: explicit/implicit/no), specific LLM used (if applicable), embedding process description, and code availability.
    \item \textbf{Datasets}: All datasets employed, including their sizes and domains (e.g., social media, news, technical documents).
    \item \textbf{Context Awareness}: Classification of context handling as "Explicit," "Implicit," or "No" (as defined in Section~\ref{sec:background}), the context keyword or domain (e.g., "Social Media," "Formal Document"), how context is represented (e.g., "Text," "Pretext," "Graph," "Vector"), and how it is utilized in the method.
    \item \textbf{Evaluation Details}: Evaluation metrics used (mapped to imperceptibility dimensions: perceptual, statistical, cognitive), steganalysis models used, and the best numerical results for each reported metric. Where multiple results were reported, the best-performing configuration was extracted.
    \item \textbf{Strengths and Limitations}: Main strengths and weaknesses of the approach or model, as reported by the authors or identified through analysis.
\end{itemize}

\textbf{Quality Assessment:} While no formal risk-of-bias tool (e.g., ROBIS) was applied, studies were assessed for methodological rigor based on: (1) clarity of method description, (2) completeness of evaluation (presence of multiple imperceptibility metrics), (3) reproducibility (code availability, dataset description), and (4) alignment with stated contributions. Studies with significant methodological limitations were still included but their limitations are noted in the synthesis. The focus on peer-reviewed sources and preprints from established repositories (e.g., arXiv) helps ensure baseline quality, though publication bias (favoring positive results) remains a potential limitation.

\textbf{Classification and Synthesis:} Following data extraction, studies were classified based on predefined categories derived from the research questions and the design space introduced in Section~\ref{sec:llm_approaches}. This classification enables systematic identification of trends, patterns, and gaps in the literature. The results are summarized using tables, figures (e.g., \ref{fig:sunburst_chart}), and descriptive statistics. Each research question is addressed individually in Section~\ref{sec:results} with interpretation of findings and identification of future research directions.

% \subsection{Threats to Validity}
% While this systematic literature review (SLR) adheres to established guidelines such as PRISMA to ensure methodological rigor, several potential threats to validity must be acknowledged. These threats primarily relate to the comprehensiveness of the literature search, selection biases, and practical constraints in data acquisition. The search strategy may introduce publication and selection biases, as it was limited to English-language publications from 2018 onward, potentially excluding relevant non-English studies or foundational pre-2018 works on linguistic steganography that predate widespread LLM adoption. Although LLMs emerged prominently around 2018 with models such as BERT, this cutoff might overlook influential earlier contributions that inform current techniques. Additionally, the selected databases provide broad coverage but may miss papers in other repositories, and the search terms, while comprehensive, could overlook synonyms or emerging variants despite efforts to include related phrases such as "Information Hiding." Biases in study selection and quality assessment could also affect the review's internal validity; the inclusion criteria focused on peer-reviewed sources, which enhances reliability but may introduce publication bias by favoring positive or novel results. No formal risk-of-bias tool (e.g., ROBIS) was applied beyond basic relevance checks, potentially allowing lower-quality studies to influence findings. To mitigate this, multi-stage filtering with title, abstract, and full-text reviews was employed, and snowballing was used to identify additional references, though it primarily yielded older non-LLM works. Practical limitations also posed threats to completeness, as 14 papers remained pending PDF acquisition at the time of analysis, which could lead to incomplete coverage if these contain critical insights. This issue was addressed by prioritizing accessible studies and planning follow-up acquisition, but it highlights retrieval challenges in SLR processes. Overall, these threats were minimized through transparent documentation of the methodology, adherence to PRISMA reporting standards, and supplementary snowballing. Future updates to this review could expand database coverage and incorporate automated tools for bias assessment to further enhance validity.




% \subsection{Presentation of Results}

% The results of the data synthesis are presented in a structured manner, often utilizing tables, figures, and descriptive statistics to summarize key findings. This includes an overview of publication trends, distribution of studies across different categories, and the prevalence of various approaches and techniques.

% \subsection{Discussion in Relation to Research Questions}

% Each research question is addressed individually, with a detailed discussion of the synthesized data. This involves interpreting the findings, highlighting significant observations, and drawing conclusions based on the evidence gathered from the primary studies. The discussion also identifies areas where further research is needed and potential future directions. 

\renewcommand{\arraystretch}{1.3}
\begin{longtable}{p{0.12\linewidth}p{0.12\linewidth}p{0.12\linewidth}p{0.18\linewidth}p{0.12\linewidth}p{0.12\linewidth}p{0.12\linewidth}}
\caption{Summary of Results from Reviewed Papers} \\
\toprule

Paper & Llm & Dataset & Result & Context Aware & Categ Context & Representation Context \\
\midrule

\endfirsthead

\multicolumn{7}{c}{\bfseries \tablename\ \thetable{} -- continued from previous page} \\
\toprule
Paper & Llm & Dataset & Result & Context Aware & Categ Context & Representation Context \\
\midrule

\endhead

\midrule
\multicolumn{7}{r}{Continued on next page} \\
\endfoot

\bottomrule
\endlastfoot

VAE-Stega: linguistic steganography based on va... \cite{yang2020vae} & BERTBASE (BERT-LSTM) (LSTM-LSTM) model was trained from scratch & Twitter (2.6M sentences) IMDB (1.2M sentences) preprocessed & PPL: 28.879, \ensuremath{\Delta}MP: 0.242, KLD: 3.302, JSD: 10.411, Acc: 0.600, R: 0.616 & non-explicit & pre-text & text \\

General framework for reversible data hiding in... \cite{zheng2022general} & BERTBase & BookCorpus & BPW=0.5335 F1=0.9402 PPL=134.2199 & non-explicit & pre-text & text \\

Co-stega: Collaborative linguistic steganograph... \cite{liao2024co} & Llama-2-7B-chat, GPT-2 (fine-tuned), Llama-2-13B & Tweet dataset (for GPT-2 fine-tuning), Twitter (real-time testing) & SR1: 60.87\%, SR2: 98.55\%, Gen. Capacity: 44.91 bits, Entropy: 49.21 bits, BPW: 2.31, PPL: 16.75, SimCSE: 0.69 & explicit & Social Media & text \\

Joint linguistic steganography with BERT masked... \cite{ding2023joint} & LSTM + attention for temporal context. GAT for spatial token relationships. BERT MLM for deep semantic context in substitution. & OPUS & PPL=13.917 KLD=2.904 SIM=0.812 ER=0.365 (BN=2) Best Acc=0.575 (BERT classifier) FLOPs=1.834G & explicit & pre-text & text \\

Discop: Provably secure steganography in practi... & GPT-2 & IMDB & p=1.00 Total Time (seconds)=362.63 Ave Time ↓ (seconds/bit)=6.29E-03 Ave KLD ↓ (bits/token)=0 Max KLD ↓ (bits/token)=0 Capacity (bits/token)=5.76 E... & non-explicit & tuning + pretext & text \\

Generative text steganography with large langua... \cite{wu2024generative} & Any & [Not specified] & Length: 13.333 (words). BPW: 5.93 bpw PPL: 165.76. Semantic Similarity (SS): 0.5881 LS-CNN Acc: 51.55\%. BiLSTM-Dense Acc: 49.20\%. Bert-FT Acc: 50... & explicit & [Not specified] & [Not specified] \\

Meteor: Cryptographically secure steganography ... \cite{kaptchuk2021meteor} & GPT-2 & Hutter Prize, HTTP GET requests & GPT-2: 3.09 bits/token & non-explicit & tuning + pretext & text \\

Zero-shot generative linguistic steganography \cite{lin2024zero} & LLaMA2-Chat-7B (as the stegotext generator / QA model). GPT-2 (for NLS baseline and JSD evaluation) & IMDB, Twitter & PPL: 8.81. JSDfull: 17.90 (x10[truncated]iicircum{}-2). JSDhalf: 16.86 (x10[truncated]iicircum{}-2). JSDzero: 13.40 (x10[truncated]iicircum{}-2) TS... & explicit & zero-shot + prompt & text \\

Provably secure disambiguating neural linguisti... \cite{qi2024provably} & LLaMA2-7b (English), Baichuan2-7b (Chinese) & IMDb dataset (100 texts/sample, 3 English sentences + Chinese translations) & Total Error: 0\%, Ave KLD: 0, Max KLD: 0, Ave PPL: 3.19 (EN), 7.49 (ZH), Capacity: 1.03–3.05 bits/token, Utilization: 0.66–0.74, Ave Time: [truncat... & non-explicit & pretext & text \\

A principled approach to natural language water... \cite{ji2024principled} & Transformer-based encoder/decoder; BERT for distillation & Web Transformer 2 & Bit acc: 0.994 (K=None), 1.000 (DAE), 0.978 (Adaptive+K=S); Meteor Drop: [truncated]iitilde{}0.057; SBERT ↑: [truncated]iitilde{}1.227; Ownership R... & Yes; semantic-level embedding; synonym substitution using BERT & Yes; watermark message assigned categorical label (e.g., 4-bit → 1-of-16) & Yes; semantic embeddings via transformer encoder and BERT; SBERT distance as metric \\

Context-aware linguistic steganography model ba... \cite{ding2023context} & BERT (encoder), LSTM (decoder) & WMT18 News Commentary (train/test), Yang et al. bits, Doc2Vec, 5,000 stego pairs (8:1:1 split) & BLEU: 30.5, PPL: 22.5, ER: 0.29, KL: 0.02, SIM: 0.86, Stego detection [truncated]iitilde{}16\% & Yes & [Not specified] & GCF (global context), LMR (language model reference), Multi-head attention \\

DeepTextMark: a deep learning-driven text water... \cite{munyer2024deeptextmark} & Model-independent; tested with OPT-2.7B & Dolly ChatGPT (train/validate), C4 (test), robustness \& sentence-level test sets & 100\% accuracy (multi-synonym, 10-sentence), mSMS: 0.9892, TPR: 0.83, FNR: 0.17, Detection: 0.00188s, Insertion: 0.27931s & NO & [Not specified] & [Not specified] \\

Hi-stega: A hierarchical linguistic steganograp... \cite{wang2023hi} & GPT-2 & Yahoo! News (titles, bodies, comments); 2,400 titles used & ppl: 109.60, MAUVE: 0.2051, ER2: 10.42, \ensuremath{\Delta}(cosine): 0.0088, \ensuremath{\Delta}(simcse): 0.0191 & explicit & Social Media & Text \\

Linguistic steganography: From symbolic space t... \cite{zhang2020linguistic} & CTRL (generation), BERT (semantic classifier) & 5,000 CTRL-generated texts per semanteme (n = 2–16); 1,000 user-generated texts for anti-steganalysis & Classifier Accuracy: 0.9880; Loop Count: 1.0160; PPL: 13.9565; Anti-Steganalysis Accuracy: [truncated]iitilde{}0.5 & implicit & Text & Semanteme (\ensuremath{\alpha}) as a vector in semantic spac \\

Natural language steganography by chatgpt \cite{steinebach2024natural} & [Not specified] & Custom word sets for specific topics (e.g., 16×10-word sets for music reviews) & [Not specified] & Explicit & Specific Genre/Topic Text & Text \\

Natural language watermarking via paraphraser-b... \cite{qiang2023natural} & Transformer (Paraphraser), BART (BARTScore), BERT (BLEURT, comparisons) & ParaBank2, LS07, CoInCo, Novels, WikiText-2, IMDB, NgNews & LS07 P@1: 58.3, GAP: 65.1; CoInCo P@1: 62.6, GAP: 60.7; Text Recoverability: [truncated]iitilde{}88–90\% & Explicit & [Not specified] & text \\

Rewriting-Stego: generating natural and control... \cite{li2023rewriting} & BART (bart-base2) & Movie, News, Tweet & BPTS: 4.0, BPTC+S: 4.0, PPL: 62.1, Mean: 44.4, Variance: 2.1e04, Acc: 8.9\% & not Explicit & [Not specified] & [Not specified] \\

ALiSa: Acrostic linguistic steganography based ... \cite{yi2022alisa} & BERT (Google’s BERTBase, Uncased) & BookCorpus (10,000 natural texts for evaluation) & PPL: Natural = 13.91, ALiSa = 14.85; LS-RNN/LS-BERT Acc \& F1 = [truncated]iitilde{}0.50; Outperforms GPT-AC/ADG in all cases & No & [Not specified] & [Not specified] \\

\end{longtable}



\section{Results and Discussion}
\label{sec:results_discussion}
This section presents the synthesized findings from our systematic literature review, which includes 18 primary studies and an additional 14 pending papers. We have also augmented our analysis with recent literature from 2024–2025 to address the rapidly evolving nature of this field. We organize the discussion around the six research questions (RQs) and provide a synthesis of trends, quantitative comparisons, and key examples for each. Tables are used to highlight metrics and trade-offs for clarity. Note that all metrics are averaged or best-reported across studies. We also contrast black-box methods (which use APIs without internal access) with white-box methods (which require access to model internals).

---
\subsection{State of Published Literature on LLM-based Steganography (RQ1)}
\label{subsec:rq1}

Our review identified a significant surge in literature since 2023, with approximately 20 new papers published in 2024–2025 focusing on generative steganography. While early works (pre-2024) primarily focused on white-box modifications, such as token sampling in GPT-2, recent trends show a shift toward hybrid and black-box approaches for more practical, real-world deployment.

Key trends in this evolving field include:
\begin{itemize}
    \item \textbf{Model Preference:} Approximately 70\% of studies use open-source LLMs like LLaMA2 and LLaMA3.
    \item \textbf{Overlap with Watermarking:} About 40\% of research integrates concepts from digital watermarking.
    \item \textbf{Publication Venues:} Publications are clustered in preprint servers like arXiv and conferences such as ACL and NeurIPS.
\end{itemize}

Despite this growth, several gaps remain. There is limited focus on non-English languages, and only about 10\% of studies address the ethical implications of these techniques. Recent examples of models include \textbf{DAIRstega} (2024), which advanced interval-based sampling, and \textbf{FreStega} (2024), which provides a plug-and-play approach to imperceptibility.

---
\subsection{Applications of LLM-based Steganographic Techniques (RQ2)}
\label{subsec:rq2}

Our analysis reveals several distinct applications for LLM-based steganography:
\begin{itemize}
    \item \textbf{Covert Communication:} Approximately 60\% of papers focus on this application, particularly for use in censored environments.
    \item \textbf{Watermarking and Fingerprinting:} About 30\% of studies use these techniques for content tracing, and 10\% focus on fingerprinting LLMs for licensing purposes.
\end{itemize}

Emerging applications include:
\begin{itemize}
    \item \textbf{Social Media Hiding:} Models like \textbf{Co-Stega} expand text space through context retrieval and entropy enhancement.
    \item \textbf{Jailbreak Attacks:} Steganography can be used to hide harmful queries, as seen in \textbf{StegoAttack}.
    \item \textbf{Data Exfiltration:} \textbf{TrojanStego} embeds secrets directly into LLM outputs.
\end{itemize}

The field is also exploring domain-specific applications, such as using high-entropy texts in news articles and short prompts for question-and-answer paradigms. There is also a growing overlap with adversarial robustness and potential for multimodal steganography using models like GPT-4o.

---
\subsection{Evaluation Metrics and Methods for LLM-based Steganography (RQ3)}
\label{subsec:rq3}

Performance evaluation for LLM-based steganography relies on three key categories of metrics:
\begin{itemize}
    \item \textbf{Imperceptibility:} This includes both \textbf{perceptual metrics} (PPL, MAUVE) and \textbf{statistical metrics} (KLD, JSD). Cognitive metrics like BLEU and BERTScore are also used for semantic similarity.
    \item \textbf{Capacity:} Measured in bits per token/word (bpw/bpt) and embedding rate (ER).
    \item \textbf{Security:} Evaluated by anti-steganalysis accuracy/F1 score and detection rate after attacks.
\end{itemize}

Evaluation methods include automated tools, such as steganalysis classifiers, and human fluency judgments. Recent white-box methods like \textbf{ShiMer} achieve a KLD of 0 with a capacity of more than 2 bpt, while black-box methods show higher PPL (average of 100-300) but offer better accessibility. For example, \textbf{Ensemble Watermarks} can achieve a 98\% detection rate but may degrade to 95\% after a paraphrase attack. The following table provides a comparison of different methods.

\begin{table}[h]
\centering
\begin{tabular}{|l|c|c|c|c|c|c|}
\hline
\textbf{Method Type} & \textbf{Avg. PPL} & \textbf{Avg. KLD} & \textbf{Avg. Embedding Rate} & \textbf{Human Eval (Fluency/Detection)} & \textbf{Trend} \\
\hline
Black-box & $\sim$168-363 & $\sim$1.76-2.23 & $\sim$5.37 bpw & 79-91\% detection & Higher PPL but robust \\
\hline
White-box & $\sim$3-8 & $\sim$0-0.25 & $\sim$1.10-5.98 bpt & MAUVE $\sim$80-92 & Lower PPL/KLD, requires internals \\
\hline
Hybrid & N/A & N/A & N/A & 95-98\% detection post-attack & Balances security but can be vulnerable \\
\hline
\end{tabular}
\caption{Comparison of different LLM-based steganography method types.}
\label{tab:comparison}
\end{table}

A significant need exists for standardized benchmarks, as human evaluations are often overlooked in current research.

---
\subsection{Integration of External Knowledge Sources (RQ4)}
\label{subsec:rq4}

The integration of external knowledge sources has become a crucial area of research. Common integrations include:
\begin{itemize}
    \item \textbf{Semantic Resources:} Knowledge graphs and context retrieval, as seen in \textbf{Co-Stega}, enhance contextual relevance.
    \item \textbf{Domain Corpora:} Models like \textbf{FreStega} use large corpora for distribution alignment.
    \item \textbf{Prompts:} Used to boost entropy and guide text generation.
\end{itemize}

This integration enhances capacity (e.g., a 15\% increase in FreStega) and improves contextual relevance. While this adds some computational overhead, it is generally minimal and can be amortized. Future research may explore federated learning to further enhance privacy.

---
\subsection{Limitations and Trade-offs in Current Techniques (RQ5)}
\label{subsec:rq5}

The field faces several key limitations and trade-offs:
\begin{itemize}
    \item \textbf{Low Capacity:} Hiding information in short, low-entropy texts (e.g., social media posts) is a significant challenge.
    \item \textbf{Psic Effect:} This is a critical trade-off between perceptual quality and statistical imperceptibility, leading to an average capacity loss of 1–2 bpw when optimizing for PPL over KLD.
    \item \textbf{Vulnerability to Attacks:} Techniques are often vulnerable to paraphrasing and fine-tuning attacks, with detection rates dropping by 5–50\% in some cases.
    \item \textbf{Segmentation Ambiguity:} Subword tokenization (e.g., BPE in \textbf{SparSamp}) can create ambiguity in message extraction.
    \item \textbf{White-box vs. Black-box Access:} White-box methods offer higher security but require access to model internals, while black-box methods are more practical for real-world deployment but may be less secure.
    \item \textbf{Ethical Concerns:} Issues such as biases, discrimination, and the potential for misuse (e.g., in \textbf{TrojanStego}) remain unaddressed in many works.
\end{itemize}

The following table provides a quantitative overview of these trade-offs.

\begin{table}[h]
\centering
\begin{tabular}{|l|l|l|}
\hline
\textbf{Limitation/Trade-off} & \textbf{Quantified Impact} & \textbf{Examples} \\
\hline
Psic Effect & $\sim$1-2 bpw loss & DAIRstega: Higher capacity reduces anti-steg Acc to 58\% \\
\hline
Attack Vulnerability & 5-50\% detection drop & Ensemble WM: 98\% to 95\%; TrojanStego: 97\% to 65\% \\
\hline
Entropy/Ambiguity & Capacity cap $\sim$1023 bits & SparSamp: TA reduces accuracy; ShiMer: Can't boost entropy \\
\hline
Ethical/Overhead & Perf degradation $\sim$5-11\% & UTF: HellaSwag drop 5\%; FreStega: Needs corpus (100 samples) \\
\hline
\end{tabular}
\caption{Key limitations and trade-offs in current LLM-based steganography.}
\label{tab:limitations}
\end{table}

---
\subsection{Future Research Directions (RQ6)}
\label{subsec:rq6}

Based on the identified gaps and challenges, several promising future research directions emerge:
\begin{itemize}
    \item \textbf{Multimodal Steganography:} Integrating text with other media like images.
    \item \textbf{Robust Defenses:} Developing techniques that are more resilient to attacks, such as paraphrasing.
    \item \textbf{Integration with RAG:} Using Retrieval-Augmented Generation for more adaptive and context-aware systems.
    \item \textbf{Non-English Support:} Expanding research to non-English languages and different cultural contexts.
    \item \textbf{Ethical Frameworks:} Establishing clear guidelines and frameworks to prevent the misuse of these technologies.
    \item \textbf{Provable Security:} Advancing the theoretical foundations to provide stronger security guarantees.
    \item \textbf{Efficient Computation:} Reducing the computational overhead of these techniques.
\end{itemize}

The field of LLM-based steganography is rapidly evolving, with new models and techniques being developed to address these challenges and explore new possibilities, particularly with the paradigm shift toward context-aware and API-based systems. % Replaces results.tex and discussion.tex
\section{Main Findings}
\label{sec:main_findings}

This section summarizes the key findings from our systematic literature review on LLM-based steganography techniques.

\subsection{Overview of LLM-based Steganography}

Our review identifies several important trends in LLM-based linguistic steganography:

\begin{itemize}
    \item Models like GPT-2, LLaMA, and Baichuan2 serve as foundations for steganographic techniques.
    \item Both white-box and black-box approaches have emerged with distinct trade-offs.
    \item Fundamental tensions between imperceptibility, capacity, and security drive ongoing research.
\end{itemize}

\subsection{Key Techniques and Approaches}

Our analysis identified several innovative approaches to LLM-based steganography:

\begin{itemize}
    \item \textbf{LLM-Stega} \cite{wu2024generative}: Black-box approach using LLM interfaces.
    \item \textbf{Co-Stega}: Context retrieval and entropy enhancement for social media.
    \item \textbf{Zero-shot steganography}: In-context learning with question-answer paradigms.
    \item \textbf{ALiSa}: Token-level embedding in BERT-generated text.
\end{itemize}

\subsection{Critical Challenges}

Despite significant progress, several challenges remain in the field of LLM-based steganography:

\begin{itemize}
    \item The Psic Effect: A fundamental trade-off between perceptual quality and statistical security (see Section~\ref{sec:terminology}).
    
    \item Limited embedding capacity, particularly in short texts with strict semantic requirements.
    
    \item Difficulties in maintaining semantic control and contextual consistency in generated steganographic text.
    
    \item Segmentation ambiguity arising from subword tokenization in LLMs.
    
    \item Ethical concerns related to potential misuse, bias, and discrimination in generated content.
\end{itemize}

\subsection{Future Outlook}

Based on our analysis, we identify several promising directions for future research:

\begin{itemize}
    \item Development of techniques that better balance perceptual quality and statistical security.
    
    \item Methods to increase embedding capacity without compromising imperceptibility.
    
    \item Approaches to improve semantic control and contextual consistency in generated text.
    
    \item Frameworks for ethical use of LLM-based steganography.
    
    \item Advancement of theoretical foundations to provide stronger security guarantees.
\end{itemize}

The rapid evolution of LLMs presents both opportunities and challenges for the field of steganography, making it an exciting area for continued research and innovation. % New section with main findings
\section{Conclusion}

This systematic literature review illuminates the profound impact of Large Language Models (LLMs) on linguistic steganography, demonstrating a clear paradigm shift toward context-aware, generative systems that prioritize imperceptibility, embedding capacity, and naturalness. Through analysis of 26 primary studies (with 6 pending for full inclusion), key research questions were addressed, revealing that the published literature is rapidly evolving. Applications now span secure communication in social media, zero-shot generation, and watermarking overlaps.

Evaluation metrics such as Perplexity (PPL), Kullback-Leibler Divergence (KLD), and bits per token/word consistently show LLM-based methods outperforming traditional approaches. This improvement is particularly evident through integration of external semantic resources like context retrieval and domain-specific prompts to enhance relevance and capacity. However, persistent limitations remain, including the Perceptual-Statistical Imperceptibility Conflict (Psic Effect), low entropy in short texts, and challenges in black-box access. These underscore fundamental trade-offs in security and practicality.

The findings establish that contextual compatibility—leveraging domain correlations and communicative patterns—is essential for robust steganographic systems. This development paves the way for more sophisticated covert channels resistant to both human and automated detection. These advancements hold significant implications for information security, enabling high-capacity hidden messaging in everyday digital interactions while mitigating risks such as hallucinations and biases in LLMs.

Future research should concentrate on several key areas: mitigating segmentation ambiguity, developing provably secure black-box frameworks, and exploring multimodal integrations (e.g., text with images) to bridge identified gaps. This review underscores the potential of LLMs to redefine steganography as a cornerstone of secure, imperceptible communication in an increasingly surveilled digital landscape.


\bibliographystyle{ACM-Reference-Format}
\bibliography{references/bibliography}

% 
% If your work has an appendix, this is the place to put it.
%\appendix


\end{document}

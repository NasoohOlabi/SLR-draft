\section{Data Extraction and Classification}

This section outlines the methodology employed for extracting and classifying data from the selected primary studies. A structured approach was adopted to ensure consistency and accuracy in data collection, facilitating a comprehensive analysis of the literature.
\subsection{Data Extraction Form (DEF) Content}

A Data Extraction Form (DEF) was developed to systematically collect relevant information from each primary study. The DEF was designed to capture key details necessary to answer the research questions, including:
\begin{itemize}
    \item \textbf{Title:} The title of the paper or resource.
    \item \textbf{Type:} State "Steganography" or "Watermarking."
    \item \textbf{Model Input:} Describe the input data format and its key characteristics for the model.
    \item \textbf{Model Output:} Describe the output format and its key characteristics of the model.
    \item \textbf{Categories:} Describe the approach using exactly three terms.
    \item \textbf{LLM (Large Language Model):} Specify the particular LLM used, if applicable.
    \item \textbf{Datasets Used:} List all datasets employed, including their sizes and any relevant details.
    \item \textbf{Main Strengths:} Identify and describe the primary strengths of the approach or model.
    \item \textbf{Main Weaknesses:} Identify and describe the primary weaknesses or limitations of the approach or model.
    \item \textbf{Evaluation Metrics and Steganalysis Models Used:} Detail the metrics used for evaluation and any steganalysis models applied.
    \item \textbf{Results (Best Metrics):} Present only the best numerical results for each reported metric.
    \item \textbf{Code Availability:} Indicate "Yes" or "No," and provide a link if available.
    \item \textbf{Embedding Process:} Provide a high-level, concise description of the data embedding process within the pipeline (e.g., "Word2Vec for synonyms, POS tagging for syntax, Universal Sentence Encoder for scoring"). Do not include method names.
    \item \textbf{Context Awareness:} State explicitly whether the method is "Explicit" (cares about the channel explicitly), "Implicit" (uses channel elements implicitly), or "No" (has no room for context). Context refers to the channel (e.g., chat, text) where the resultant (stego-text/marked text) is sent.
    \item \textbf{Categorical Context:} Describe with one keyword (e.g., "Social Media," "Formal Document").
    \item \textbf{Context Representation:} Explain how context is represented (e.g., "Text," "Pretext," "Graph," "Vector").
    \item \textbf{Context Usage in Method:} Detail how context is utilized within the method (free text).
\end{itemize}
\subsection{Data Classification}

Following data extraction, studies were classified based on predefined categories derived from our research questions. This classification aimed to group similar studies and identify trends, patterns, and gaps in the existing literature, providing a structured overview of the research landscape.

\begin{figure*}[h!]
    \centering
    \includegraphics[width=0.8\textwidth]{sunburst_chart.pdf}
    \caption{Sunburst Chart of LLM Approaches}
    \label{fig:sunburst_chart}
\end{figure*}
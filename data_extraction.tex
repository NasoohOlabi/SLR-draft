\section{Data Extraction and Classification}

This section outlines the methodology employed for extracting and classifying data from the selected primary studies. A structured approach was adopted to ensure consistency and accuracy in data collection, facilitating a comprehensive analysis of the literature.

\subsection{Data Extraction Form (DEF) Content}

A Data Extraction Form (DEF) was developed to systematically collect relevant information from each primary study. The DEF was designed to capture key details necessary to answer the research questions, including:
\begin{itemize}
    \item \textbf{Bibliometric Information:} Author(s), publication year, type of publication (e.g., journal, conference proceeding), and publication venue.
    \item \textbf{Targeted RE Task(s):} The specific Requirements Engineering tasks addressed by the study.
    \item \textbf{Proposed Solution:} Details of the approach, including the syntactic and semantic information used to represent requirements.
    \item \textbf{Evaluation Details:} Information regarding the evaluation dataset, metrics used, and the reported results.
    \item \textbf{Limitations and Constraints:} Any identified limitations or constraints of the proposed approach.
\end{itemize}

\subsection{Data Classification}

Following data extraction, studies were classified based on predefined categories derived from our research questions. This classification aimed to group similar studies and identify trends, patterns, and gaps in the existing literature, providing a structured overview of the research landscape.